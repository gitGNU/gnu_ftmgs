%-------------------------------------------------------------------------------
% Copyright (C)  2012  Vicente Benjumea, University of Malaga, Spain
% 
% Redistribution and use in source (LaTeX) and 'compiled' forms (SGML,
% HTML, PDF, PostScript, RTF and so forth) with or without
% modification, are permitted provided that the following conditions
% are met:
% 
%  1. Redistributions of source code (LaTeX) must retain the
%     above copyright notice, this list of conditions and the following
%     disclaimer as the first lines of this file unmodified.
% 
%  2. Redistributions in compiled form (transformed to other DTDs,
%    converted to PDF, PostScript, RTF and other formats) must
%    reproduce the above copyright notice, this list of conditions and
%    the following disclaimer in the documentation and/or other
%    materials provided with the distribution.
% 
% THIS DOCUMENTATION IS PROVIDED BY THE COPYRIGHT HOLDERS "AS IS" AND
% ANY EXPRESS OR IMPLIED WARRANTIES, INCLUDING, BUT NOT LIMITED TO,
% THE IMPLIED WARRANTIES OF MERCHANTABILITY AND FITNESS FOR A
% PARTICULAR PURPOSE ARE DISCLAIMED. IN NO EVENT SHALL THE COPYRIGHT
% HOLDERS BE LIABLE FOR ANY DIRECT, INDIRECT, INCIDENTAL, SPECIAL,
% EXEMPLARY, OR CONSEQUENTIAL DAMAGES (INCLUDING, BUT NOT LIMITED TO,
% PROCUREMENT OF SUBSTITUTE GOODS OR SERVICES; LOSS OF USE, DATA, OR
% PROFITS; OR BUSINESS INTERRUPTION) HOWEVER CAUSED AND ON ANY THEORY
% OF LIABILITY, WHETHER IN CONTRACT, STRICT LIABILITY, OR TORT
% (INCLUDING NEGLIGENCE OR OTHERWISE) ARISING IN ANY WAY OUT OF THE
% USE OF THIS DOCUMENTATION, EVEN IF ADVISED OF THE POSSIBILITY OF
% SUCH DAMAGE.
%-------------------------------------------------------------------------------
\documentclass[a4paper]{article}
%%\usepackage{latexsym}
%%\usepackage{amssymb}
%%\usepackage{amsfonts}
%%\usepackage{multicol}
\usepackage{alltt}
%-------------------------------------------------------------------------------
\newenvironment{code}%
{\begin{quote}\footnotesize\begin{alltt}}%
{\end{alltt}\end{quote}}%
%---------------------------------------
\newenvironment{api}%
{\noindent$\bullet$\hfill\begin{minipage}[t]{0.97\linewidth}\footnotesize\begin{alltt}}%
{\end{alltt}\end{minipage}}%
%---------------------------------------
\newenvironment{apix}%
{\noindent$\diamond$\hfill\begin{minipage}[t]{0.97\linewidth}\footnotesize\begin{alltt}}%
{\end{alltt}\end{minipage}}%
%-------------------------------------------------------------------------------
\newcommand{\tuple}[1]{\ensuremath{\langle #1 \rangle}}
%-------------------------------------------------------------------------------
\title{FTMGS: Fair Traceable Multi-Group Signatures}
\author{User Manual (DRAFT)}
\date{Version 0.1}
\bibliographystyle{abbrv}
%-------------------------------------------------------------------------------
%-------------------------------------------------------------------------------
\begin{document}
\maketitle
\enlargethispage{8mm}
\tableofcontents
%-------------------------------------------------------------------------------
\newpage
\vspace*{\fill}\noindent
\begin{minipage}{\linewidth}\footnotesize
\noindent
Copyright (c)  2012  Vicente Benjumea, University of Malaga, Spain\\

\noindent
Redistribution and use in source (LaTeX) and 'compiled' forms
(SGML, HTML, PDF, PostScript, RTF and so forth) with or without
modification, are permitted provided that the following conditions are
met:
\begin{enumerate}
\item Redistributions of source code (LaTeX) must retain the
      above copyright notice, this list of conditions and the
      following disclaimer as the first lines of this file unmodified.

\item Redistributions in compiled form (transformed to other DTDs,
      converted to PDF, PostScript, RTF and other formats) must
      reproduce the above copyright notice, this list of conditions
      and the following disclaimer in the documentation and/or other
      materials provided with the distribution.
\end{enumerate}
THIS DOCUMENTATION IS PROVIDED BY THE COPYRIGHT HOLDERS "AS IS" AND
ANY EXPRESS OR IMPLIED WARRANTIES, INCLUDING, BUT NOT LIMITED TO, THE
IMPLIED WARRANTIES OF MERCHANTABILITY AND FITNESS FOR A PARTICULAR
PURPOSE ARE DISCLAIMED. IN NO EVENT SHALL THE COPYRIGHT HOLDERS BE
LIABLE FOR ANY DIRECT, INDIRECT, INCIDENTAL, SPECIAL, EXEMPLARY, OR
CONSEQUENTIAL DAMAGES (INCLUDING, BUT NOT LIMITED TO, PROCUREMENT OF
SUBSTITUTE GOODS OR SERVICES; LOSS OF USE, DATA, OR PROFITS; OR
BUSINESS INTERRUPTION) HOWEVER CAUSED AND ON ANY THEORY OF LIABILITY,
WHETHER IN CONTRACT, STRICT LIABILITY, OR TORT (INCLUDING NEGLIGENCE
OR OTHERWISE) ARISING IN ANY WAY OUT OF THE USE OF THIS DOCUMENTATION,
EVEN IF ADVISED OF THE POSSIBILITY OF SUCH DAMAGE.
\end{minipage}
\newpage
%-------------------------------------------------------------------------------
\section{Introduction}
%-------------------------------------------------------------------------------

The FTMGS library implements the \emph{Fair Traceable Multi-Group
  Signatures} scheme as defined in~\cite{bcly:2008:ftmgs}.

Fair Traceable Multi-Group Signatures (FTMGS, pronounced: fat-mugs),
is a primitive that supports anonymity with extended concerns that
rise in realistic scenarios. It can be regarded as a primitive that
has the flavor of anonymous signatures with various revocations but
with a refined notion of access control (via multiple groups) and thus
supporting anonymous activities in a fashion similar to anonymous
credential systems. The main issues that make this primitive suitable
to various trust relationships are:

\begin{itemize}%
\item 
The group manager (GM) creates and manages a group with the help of
some Fairness Authorities (FA), which are only involved when special
circumstances arise.

\item 
Users (U) join a group, and become members (M), if the group manager
(GM) allows them to do so.

\item 
Members (M) issue group signatures (on behalf of the whole group)
which can be verified with the \emph{group public key} (GPK). These
group signatures provide the guarantee the their issuers are indeed
members of the group, but they are anonymous and unlinkable, in the
sense that there is no \emph{direct} way to identify which member
issued a given signature, nor it is even possible to link different
signatures as having been issued by the same member of the group.

\item
Belonging to a group usually implies that members fulfill the set of
privileges required by the group manager (GM) to join the group.

In authorization and access control scenarios, issuing a group
signature is a suitable way to prove, in an anonymous and unlinkable
way, that the issuer fulfills a set of privileges required by the GM
to join the group.

Sometimes, in authorization scenarios, a user must simultaneously
prove that is a member of several groups at tehsame time, and thus she
fulfills the privileges to belong to all these groups, therefore she
must issue group signatures for all of them, and at the same time she
must prove that all these signatures were issued by the same real
anonymous user, that is, they have not been issued by a collusion of
different users belonging to different groups.

\item 
It includes multi-group features to guarantee that several signatures
have been issued by the same anonymous user with no detriment of
user’s anonymity. This allows limited local linkability most useful in
many cases (linking is user controlled).

\item 
It includes a mechanism to dissuade the group members from sharing
their private membership keys. This is very useful in increasing the
incentive for better “access control” to anonymous credentials.

\item 
In the undesirable case of abuse of anonymity, the group manager in
collaboration with the fairness authorities provide a mechanism to
identify which member of the group actually issued a given group
signature, breaking in this way with the anonymity of the misbehaved
user.

Note that this breaking of anonymity can only be done if all the
fairness authorities agree on doing so, because there are enough
circumstances that motivate such action. However, if any of the
fairness authorities thinks there are not enough reasons for that,
then anonymity can not be broken. In this way, the fairness
authorities become the guarantee that anonymity will be only broken
when there are enough reasons to do so.

\item
On the other way, if a member is under suspicion, it is possible to
obtain, again if the fairness authorities agree, a \emph{member
  tracing key} that allows to identify the signatures that were issued
by such a member of the group under suspicion. Again, the fairness
authorities become the guarantee that anonymity will be only broken
when there are enough reasons to do so.

\item
This previous mechanism can also be used to revoke membership from the
group for a given member.

\item 
Note that a single fairness authority alone cannot do the opening or
revealing. In this way, a user’s sensitive information can be
guaranteed only to be disclosed when there exist enough reasons.

\end{itemize}%


%-------------------------------------------------------------------------------
\section{Operational Model}
%-------------------------------------------------------------------------------

The group manager creates a group with the collaboration of designated
fairness authorities. A user, that has been authorized by some
external procedure, is able to join the group by engaging in an
interactive protocol with the group manager. The external user’s
authentication can be based on her identity (DSA signature) or even an
anonymous authentication supported by this new primitive (FTMGS
signature). At the end of the procedure, the group manager gets some
sensitive data regarding the new member (i.e. join transcript with
authentication information), and the user gets a membership private
key that enables her to issue signatures on behalf of the group.

When a user wants to carry out a transaction with a server, she
sometimes has to generate a proof to show she has the required
privilege. This proof usually implies that she belongs to several
groups. In this case, she issues suitable signatures for the involved
groups, and establishes a link among them to guarantee that they have
been issued by the same single anonymous user. This proof is anonymous
and unlinkable.

Under critical circumstances, fairness authorities and the judge open
a signature to identify a malicious user. If necessary, they may also
reveal her tracing trapdoor so that tracing agents, using the
trapdoor, trace all the transactions she issued.

The user owns a single master key, and this key is embedded, when
joining to the group, in every membership private key of hers. Because
this master key is actually the private key corresponding to her
public key (e.g., her DSA public key published via the PKI), she is
dissuaded from sharing her membership private keys. Moreover, this
binding also guarantees that different users have different master
keys.

This master key provides a common nexus among all membership private
keys that belong to each user, so that she can link any two signatures
of hers by proving that the signatures have been issued by membership
private keys into which the same master key is embedded. This
capability of linking helps our scheme to enjoy multi-group
features. Note that even when the user joins the group by means of an
anonymous authentication, the join procedure forces her to use the
same master key, so that the relationship between her master key and
public key still holds.

The group is created by the collaboration among the group manager and
the fairness authorities.  The GM is able to join new
members. Fairness authorities are also involved in the setup process,
in such a way that the keys related to opening and revealing are
distributed among the fairness authorities. Therefore, opening a
signature or revealing a member tracing key requires the agreement and
participation of fariness authorities.

Opening a signature is a matter of the distributed decryption, by the
fairness authorities, of part of the signature. Likewise, revealing a
member tracing key is also a matter of the distributed decryption, by
the fairness authorities, of the encrypted member's tracing key.

Finally, the join transcript also holds some non-repudiable proofs
that allow to verify the integrity of the record, making the scheme
robust against some kind of database manipulation.

%-------------------------------------------------------------------------------
\section{Sequence Diagrams}\label{seqdia}
%-------------------------------------------------------------------------------
\newcommand{\man}{\begin{tabular}{@{}c@{}}{$\circ$}\\$/|\backslash$\\$/\backslash$\\\end{tabular}}
\newcommand{\boxed}[1]{\begin{tabular}{|c|}\hline#1\\\hline\end{tabular}}
\newcommand{\mcol}[1]{\multicolumn{2}{c}{\begin{tabular}{c}#1\end{tabular}}}
\newcommand{\rarrowhead}{\hspace*{\fill}\begin{picture}(0,0)(-2,5.5)\texttt{>}\end{picture}}
\newcommand{\larrowhead}{\begin{picture}(0,0)(7,5.5)\texttt{<}\end{picture}\hspace*{\fill}}
\newcommand{\msg}[1]{{\sffamily\scriptsize$\langle$ {#1} $\rangle$}}

%-------------------------------------------------------------------------------
\subsection{Group Setup}\label{seqdia:group}
%-------------------------------------------------------------------------------
\noindent{\footnotesize\begin{tabular}{c|ccc|ccc|ccc|c}
\mcol{\man\\GM}&&\mcol{\man\\FA$_1$}&{\Large$\cdots$}&\mcol{\man\\FA$_n$}&&\mcol{\man\\FA$_0$}\\
\hspace*{10ex}&&&\hspace*{9ex}&&&\hspace*{9ex}&&&\hspace*{7ex}&\\
&&&&&&&&&\mcol{\boxed{FA0-Setup}}\\
&&&&\larrowhead&&&\larrowhead&\msg{fapk}&&\\\cline{5-10}
&&&&&&&&&&\\
&&&\mcol{\boxed{FAj-Setup}}&{\Large$\cdots$}&\mcol{\boxed{FAj-Setup}}&&&\\
&&&&&&&&&&\\
\mcol{\boxed{GM-Init-Setup}}&&&&&&&&&\\
&&\msg{gpk}&\rarrowhead&&&\rarrowhead&&&&\\\cline{2-7}
&&&&&&&&&&\\
&&&\mcol{\boxed{FAj-GrSetup}}&{\Large$\cdots$}&\mcol{\boxed{FAj-GrSetup}}&&&\\
&\larrowhead&\msg{fa$_1$pk}&&&&&&&&\\\cline{2-4}
&&&&&&&&&&\\
\mcol{\boxed{GM-GrSetup}}&&&&&&&&&\\
&&&&&&&&&&\\
\mcol{\Large$\vdots$}&&&&{\Large$\ddots$}&&&&&\\
&\larrowhead&\msg{fa$_n$pk}&&&&&&&&\\\cline{2-7}
&&&&&&&&&&\\
\mcol{\boxed{GM-GrSetup}}&&&&&&&&&\\
&&&&&&&&&&\\
\end{tabular}}

%-------------------------------------------------------------------------------
\subsubsection*{Check GPK}\label{seqdia:check}
%-------------------------------------------------------------------------------
\noindent{\footnotesize\begin{tabular}{c|c}
\mcol{\man\\Any}\\
\hspace*{11ex}&\\
\mcol{\boxed{Check-GPK}}\\
&\\
\mcol{\boxed{Check-GPK-FA$_1$}}\\
&\\
\mcol{\Large$\vdots$}\\
&\\
\mcol{\boxed{Check-GPK-FA$_n$}}\\
&\\
\end{tabular}}

%-------------------------------------------------------------------------------
\subsection{Join New Member}\label{seqdia:join}
%-------------------------------------------------------------------------------
\noindent{\footnotesize\begin{tabular}{c|ccc|c}
\mcol{\man\\User}&&\mcol{\man\\GM}\\
\hspace*{11ex}&&&\hspace*{11ex}&\\
&\larrowhead&{\sffamily\scriptsize\boxed{External User Authorization (DSA/FTMGS)}}&\rarrowhead&\\\cline{2-4}
&&&&\\
\mcol{\boxed{Extract-UMK\\from-DSA/MSK}}&&\mcol{\boxed{Extract-UAUTH\\from-DSA/SG}}\\
&&&&\\
\mcol{\boxed{Join-Usr}}&&&\\
&&\msg{pbl:u1}&\rarrowhead&\\\cline{2-4}
&&&&\\
&&&\mcol{\boxed{Join-GM}}\\
&\larrowhead&\msg{pbl:gm2}&&\\\cline{2-4}
&&&&\\
\mcol{\boxed{Join-Usr}}&&&\\
&&\msg{pbl:u3}&\rarrowhead&\\\cline{2-4}
&&&&\\
&&&\mcol{\boxed{Join-GM}}\\
&\larrowhead&\msg{pbl:gm4}&&\\\cline{2-4}
&&&&\\
\mcol{\boxed{Join-Usr}}&&&\\
&&&&\\
\end{tabular}}

%-------------------------------------------------------------------------------
\subsection{Sign / Verify}\label{seqdia:sign}
%-------------------------------------------------------------------------------
\noindent{\footnotesize\begin{tabular}{c|ccc|c}
\mcol{\man\\Member}&&\mcol{\man\\Any}\\
\hspace*{4ex}&&&\hspace*{5ex}&\\
\mcol{\boxed{Sign}}&&&\\
&&\msg{msg, signature}&\rarrowhead&\\\cline{2-4}
&&&&\\
&&&\mcol{\boxed{Verify}}\\
&&&&\\
\end{tabular}}

%-------------------------------------------------------------------------------
\subsection{Open / Check}\label{seqdia:open}
%-------------------------------------------------------------------------------
\noindent{\footnotesize\begin{tabular}{c|ccc|ccc|ccc|c}
\mcol{\man\\Judge}&&\mcol{\man\\FA$_1$}&{\Large$\cdots$}&\mcol{\man\\FA$_n$}&&\mcol{\man\\GM}\\
\hspace*{7ex}&&&\hspace*{9ex}&&&\hspace*{9ex}&&&\hspace*{9ex}&\\
&&\msg{sg}&\rarrowhead&&&\rarrowhead&&&&\\\cline{2-7}
&&&&&&&&&&\\
&&&\mcol{\boxed{Open-DShare}}&{\Large$\cdots$}&\mcol{\boxed{Open-DShare}}&&&\\
&\larrowhead&\msg{dshare$_1$}&&&&&&&&\\\cline{2-4}
&&&&&&&&&&\\
\mcol{\boxed{Open-Sign}}&&&&&&&&&\\
&&&&&&&&&&\\
\mcol{\Large$\vdots$}&&&&{\Large$\ddots$}&&&&&\\
&\larrowhead&\msg{dshare$_n$}&&&&&&&&\\\cline{2-7}
&&&&&&&&&&\\
\mcol{\boxed{Open-Sign}}&&&&&&&&&\\
&&\msg{op}&&&&&&&\rarrowhead&\\\cline{2-10}
&&&&&&&&&&\\
&&&&&&&&&\mcol{\boxed{Open-Check\\+ DB-Search}}\\
&\larrowhead&\msg{usr-id}&&&&&&&&\\\cline{2-10}
&&&&&&&&&&\\
\end{tabular}}

%-------------------------------------------------------------------------------
\subsection{Reveal / Trace}\label{seqdia:reveal}
%-------------------------------------------------------------------------------
\hspace*{-5em}\noindent{\footnotesize\begin{tabular}{c|ccc|ccc|ccc|ccc|c}
\mcol{\man\\Judge}&&\mcol{\man\\GM}&&\mcol{\man\\FA$_1$}&{\Large$\cdots$}&\mcol{\man\\FA$_n$}&&\mcol{\man\\TA}\\
\hspace*{10ex}&&&\hspace*{7ex}&&&\hspace*{9ex}&&&\hspace*{9ex}&&&\hspace*{4ex}&\\
&&\msg{usr-id}&\rarrowhead&&&&&&&&&&\\\cline{2-4}
&&&&&&&&&&&&&\\
&&&\mcol{\boxed{DB-Search}}&&&&&&&&&\\
&\larrowhead&\msg{mr}&&&&&&&&&&&\\\cline{2-4}
&&&&&&&&&&&&&\\
&&\msg{mr}&&&&\rarrowhead&&&\rarrowhead&&&&\\\cline{2-10}
&&&&&&&&&&&&&\\
&&&&&&\mcol{\boxed{Reveal-DShare}}&{\Large$\cdots$}&\mcol{\boxed{Reveal-DShare}}&&&\\
&\larrowhead&\msg{dshare$_1$}&&&&&&&&&&&\\\cline{2-7}
&&&&&&&&&&&&&\\
\mcol{\boxed{Reveal-MTKey}}&&&&&&&&&&&&\\
&&&&&&&&&&&&&\\
\mcol{\Large$\vdots$}&&&&&&&{\Large$\ddots$}&&&&&\\
&\larrowhead&\msg{dshare$_n$}&&&&&&&&&&&\\\cline{2-10}
&&&&&&&&&&&&&\\
\mcol{\boxed{Reveal-MTKey}}&&&&&&&&&&&&\\
&&\msg{mtkey}&&&&&&&&&&\rarrowhead&\\\cline{2-13}
&&&&&&&&&&&&&\\
&&&&&&&&&&&&\mcol{\boxed{Trace}}\\
&\larrowhead&\msg{sg$_i$}&&&&&&&&&&&\\\cline{2-13}
&&&&&&&&&&&&&\\
\end{tabular}}

%-------------------------------------------------------------------------------
\section{Implementation: Features and Security Issues}
%-------------------------------------------------------------------------------
\begin{itemize}%
\item Based on: \emph{FTMGS: Fair Traceable Multigroup
  Signatures}~\cite{bcly:2008:ftmgs}.

\item Developed in C (ANSI-C 89) and GNU/Linux (code also for Windows)
\begin{itemize}%
\item Minor system dependencies (random entropy and word endianness)
\item Easy port to other platforms
\end{itemize}%

\item Licensed under LGPLv2.1

\item It uses GMP library (LGPL) for multiple precision arithmetic

\item It uses LIBTASN1 library (LGPL) for ANS1 data conversion
\begin{itemize}%
\item In future releases, this dependency will be removed for
      efficiency purposes, and ASN1 conversion will be provided
      internally.
\end{itemize}%

\item It uses SHA library code from IETF RFC-6234 (license included in
  the \verb|licenses| directory)

\item All functions are re-entrant

\item Signatures of Knowledge follows the specified in
        Traceable Signatures~\cite{KTY04}.

\item Non-Adaptive Drawing of Random Powers follows the specified in
        Traceable Signatures~\cite{KTY04}.

\item Random Number generator based on \emph{NIST-SP-800-90} document
  (based on SHA) 
  \begin{itemize}%
  \item Entropy source at seeding is \verb|/dev/random| for
    \verb|True_Entropy|, and \verb|/dev/urandom| for
    \verb|Pseudo_Entropy| in a Linux environment, and
    \verb|CryptGenRandom| in a Windows environment.
  \item It implements Hash\_DRBG based on SHA-256, with the following
    parameters: HSStrength = 256, MinEntropy = 256, SeedLen = 440.
  \item Validation based on FIPS-140-2
  \end{itemize}%

\item Sophie-Germain Prime Numbers based on~\cite{CS00}.

\item Miller-Rabin number of tests based on FIPS-186-3 recomendations
\begin{itemize}%
\item High-Security: (modulus factors, sophie-germain primes)
\item Low-Security: (exponents for: Group: hj ; Join: ei; FA-PrivKey: xj)
\begin{center}\footnotesize
\begin{tabular}{cc}
High Security & Low Security \\
\begin{tabular}{l|cccc}\hline
Nbits  & 512 & 1024 & 2048 & $>$2048 \\\hline
Ntests &  40 &   40 &   56 & 64 \\\hline
\end{tabular}
&
\begin{tabular}{l|cccc}\hline
Nbits  & 512 & 1024 & 2048 & $>$2048 \\\hline
Ntests &   7 &    5 &    4 & 4 \\\hline
\end{tabular}
\end{tabular}
\end{center}
\end{itemize}%

\item FTMGS Security Parameters:
\begin{center}\footnotesize
\begin{tabular}{l|ccc}\hline
Nu  & 1024 & 2048 & 3072 \\\hline
K   &  128 &  256 &  512 \\\hline
\end{tabular}
\end{center}

\item Accepts DSA User Authentication in Join. DSA [1024, 2048, 3072].

\item Precomputations enabled/disabled at compile time

\item Current Development Status: Alpha

\item Things To-Do:
\begin{itemize}%
\item Change the names of API functions and types to make them shorter.
\item Define and request an ASN.1 Object Identifier (OID) number for FTMGS
\item Prepare a battery of tests
\item Accepting RSA User Authentication in Join
\item Checking that a RSA modulus lacks of small prime factors
\item Distributed generation of RSA modulus with unknown factorization
\item Improving the automatic building and installation (GNU/Linux)
\item Developing installation packages for Debian [optional]
\item Porting the building and installation to other systems (Windows)
\end{itemize}%
\end{itemize}%

%-------------------------------------------------------------------------------
\section{Performace}
%-------------------------------------------------------------------------------

\subsection*{FTMGS Modular Exponentiations}

{\footnotesize\begin{tabular}{lrccc}\hline
&& Precomp & No-Precomp & \\\hline
Join && 50 & 64 & \\
Join & (Join+UsrAuth) & 55 & 69 & \\
\\
Sign && 19 & 19 & \\
Vrfy && 22 & 28 & \\
\\
Open && NFAS$\times$9 & NFAS$\times$9 & \\
Check && 11 & 15 & [1$\times$VrfyJoinLog] \\
Check & (Join+UsrAuth) & 14 & 18 & \\
\\
Reveal && NFAS$\times$20+12 & NFAS$\times$24+16 & [(NFAS+1)$\times$VrfyJoinLog] \\
Reveal & (Join+UsrAuth) & NFAS$\times$23+15 & NFAS$\times$27+19 & \\
Trace && 1 & 1 & \\
\\
Claim && 2 & 2 & [1$\times$check(y==$g^x$)] \\
VrfyClaim && 3 & 3 & \\
\\
Link && 4 & 4 & [2$\times$check(y==$g^x$)] \\
VrfyLink && 6 & 6 & \\
\\
VrfyJoinLog && 11 & 15 & \\
VrfyJoinLog & (Join+UsrAuth) & 14 & 18 & \\
\\\hline
\end{tabular}}

\subsection*{FTMGS Timings (in seconds) [Pentium 32 bits 2GHz]}

{\footnotesize\begin{tabular}{lrccc}\hline
&& Precomp & No-Precomp & \\\hline
Join && 0.33 & 0.42 & [+ searching for prime $e_i$]\\
Join & (Join+UsrAuth) & 0.35 & 0.44 & [+ searching for prime $e_i$]\\
\\
Sign && 0.05 & 0.05 & \\
Vrfy && 0.05 & 0.08 & \\
\\
Open && NFAS$\times$0.02+0.01 & NFAS$\times$0.02+0.01 & \\
Check && 0.07 & 0.11 & [1$\times$VrfyJoinLog] \\
Check & (Join+UsrAuth) & 0.08 & 0.12 & \\
\\
Reveal && NFAS$\times$0.34+0.10 & NFAS$\times$0.39+0.13 & [(NFAS+1)$\times$VrfyJoinLog] \\
Reveal & (Join+UsrAuth) & NFAS$\times$0.35+0.10 & NFAS$\times$0.39+0.15 & \\
Trace && 0.01 & 0.01 & \\
\\
Claim && 0.01 & 0.01 & [1$\times$check(y==$g^x$)] \\
VrfyClaim && 0.01 & 0.01 & \\
\\
Link && 0.01 & 0.01 & [2$\times$check(y==$g^x$)] \\
VrfyLink && 0.01 & 0.01 & \\
\\
VrfyJoinLog && 0.07 & 0.11 & \\
VrfyJoinLog & (Join+UsrAuth) & 0.08 & 0.12 & \\
\\\hline
\end{tabular}}

\subsection*{FTMGS Timings (in seconds) [Pentium 64 bits 3.2GHz]}

{\footnotesize\begin{tabular}{lrccc}\hline
&& Precomp & No-Precomp & \\\hline
Join && 0.05 & 0.07 & \\
Join & (Join+UsrAuth) & 0.07 & 0.08 & \\
\\
Sign && 0.01 & 0.01 & \\
Vrfy && 0.01 & 0.02 & \\
\\
Open && NFAS$\times$0.0025 & NFAS$\times$0.0025 & \\
Check && 0.01 & 0.02 & [1$\times$VrfyJoinLog] \\
Check & (Join+UsrAuth) & 0.01 & 0.02 & \\
\\
Reveal && NFAS$\times$0.05 & NFAS$\times$0.05 & [(NFAS+1)$\times$VrfyJoinLog] \\
Reveal & (Join+UsrAuth) & NFAS$\times$0.05 & NFAS$\times$0.06 & \\
Trace && 0.00 & 0.00 & \\
\\
Claim && 0.00 & 0.00 & [1$\times$check(y==$g^x$)] \\
VrfyClaim && 0.00 & 0.00 & \\
\\
Link && 0.00 & 0.00 & [2$\times$check(y==$g^x$)] \\
VrfyLink && 0.00 & 0.00 & \\
\\
VrfyJoinLog && 0.01 & 0.01 & \\
VrfyJoinLog & (Join+UsrAuth) & 0.01 & 0.02 & \\
\\\hline
\end{tabular}}

% #-------------------------------------------------------------------------------
%                 Precomp                 No-Precomp
% #-------------------------------------------------------------------------------
% Join:           50 pow                  64 pow
%   Join+UsrAuth: 55 pow                  69 pow
% ----
% Sign:           19 pow                  19 pow
% Vrfy:           22 pow                  28 pow
% ----
% Open:           NFAS*9 pow              NFAS*9 pow
% Check:          11 pow                  15 pow          [1*VrfyJoinLog]
%   Join+UsrAuth: 14 pow                  18 pow
% ----
% Reveal:         NFAS*20+12 pow          NFAS*24+16 pow  [(NFAS+1)*VrfyJoinLog]
%   Join+UsrAuth: NFAS*23+15 pow          NFAS*27+19 pow
% Trace:          1 pow                   1 pow
% ----
% Claim:          2 pow                   2 pow           [1*check(y==g^x)]
% VrfyClaim:      3 pow                   3 pow
% ----
% Link:           4 pow                   4 pow           [2*check(y==g^x)]
% VrfyLink:       6 pow                   6 pow
% ----
% VrfyJoinLog:    11 pow                  15 pow
%   Join+UsrAuth: 14 pow                  18 pow
% #-------------------------------------------------------------------------------

% #-------------------------------------------------------------------------------
% #- FTMGS TIMINGS (2GHz) --------------------------------------------------------
% #-------------------------------------------------------------------------------
%                 Precomp                 No-Precomp
% #-------------------------------------------------------------------------------
% Join:           0.33 sec                0.42 sec
%   Join+UsrAuth: 0.35 sec                0.44 sec
% ----
% Sign:           0.05 sec                0.05 sec
% Vrfy:           0.05 sec                0.08 sec
% ----
% Open:           NFAS*0.02+0.01 sec      NFAS*0.02+0.01 sec
% Check:          0.07 sec                0.11 sec           [1*VrfyJoinLog]
%   Join+UsrAuth: 0.08 sec                0.12 sec
% ----
% Reveal:         NFAS*0.34+0.10 sec      NFAS*0.39+0.13 sec [(NFAS+1)*VrfyJLog]
%   Join+UsrAuth: NFAS*0.35+0.10 sec      NFAS*0.39+0.15 sec
% Trace:          0.01 sec                0.01 sec
% ----
% Claim:          0.01 sec                0.01 sec           [1*check(y==g^x)]
% VrfyClaim:      0.01 sec                0.01 sec
% ----
% Link:           0.01 sec                0.01 sec           [2*check(y==g^x)]
% VrfyLink:       0.01 sec                0.01 sec
% ----
% VrfyJoinLog:    0.07 sec                0.11 sec
%   Join+UsrAuth: 0.08 sec                0.12 sec
% #-------------------------------------------------------------------------------

%-------------------------------------------------------------------------------
\section{Installing the FTMGS Library}
%-------------------------------------------------------------------------------

\begin{itemize}%
\item To cleanup the library
\begin{verbatim}
    $ make cleanup
\end{verbatim}
\item To compile the library
\begin{verbatim}
    $ make
\end{verbatim}
\item To install the library (as root)
\begin{verbatim}
    # make install
\end{verbatim}
\item To compile the test program
\begin{verbatim}
    $ gcc -o test test.c -lftmgs
\end{verbatim}
\item To run the test program
\begin{verbatim}
	$ ./test
\end{verbatim}
\item To uninstall the library (as root)
\begin{verbatim}
    # make uninstall
\end{verbatim}
\end{itemize}%

Please, see the README file for more updated information

%-------------------------------------------------------------------------------
\section{Compiling and Linking with the FTMGS Library}
%-------------------------------------------------------------------------------

Source files the make use of the facilities provided by the FTMGS
library should include the header file \verb|ftmgs.h|, where the
public API is defined. This header file is installed in the system
include directory by the installation process, thus the file should be
included by the following directive:
\begin{verbatim}
    #include <ftmgs.h>
\end{verbatim}

The object files should be linked with the FTMGS library
(\verb|libftmgs.a| or \verb|libftmgs.so| in unix like systems):
\begin{verbatim}
    $ gcc -o test test.c -lftmgs
\end{verbatim}

%-------------------------------------------------------------------------------
\section{Abstract Data Types and Function API}
%-------------------------------------------------------------------------------

%-------------------------------------------------------------------------------
\subsection{Miscellaneous Data Types}
%-------------------------------------------------------------------------------

The public API of FTMGS library is defined in the header file
\verb|ftmgs.h|, which should be included.

%---------------------------------------
\subsubsection*{FTMGS Version and Revision}
%---------------------------------------

FTMGS version and revision numbers are defined as preprocessor macros,
to allow their use from both, C programs and preprocessor
conditionals. They follow the following guidelines:

{\footnotesize\begin{itemize}%
\item Version Major: identifies changes in public API and functionality
\item Version Minor: identifies internal changes that do not affect the public behavior
\item Revision: for bug fixes
\end{itemize}}%
They are defined as follows, where a \emph{version major} value lower
than 1 identifies alpha or beta pre-releases.
\begin{code}
#define FTMGS_VERSION_MAJOR     0
#define FTMGS_VERSION_MINOR     1
#define FTMGS_REVISION          0
\end{code}

%---------------------------------------
\subsubsection*{Boolean Type}
%---------------------------------------

The boolean type and values are defined under preprocessor
conditionals to avoid clashing with other definitions. Additionally,
the boolean values are checked for their right definitions.

\begin{code}
#ifndef BOOL_T_DEFINED__
#define BOOL_T_DEFINED__    1
typedef char bool_t;
#endif
#ifndef TRUE
#define TRUE    1
#elif TRUE == 0
#error "Bad definition of symbol TRUE"
#endif
#ifndef FALSE
#define FALSE   0
#elif FALSE != 0
#error "Bad definition of symbol FALSE"
#endif
\end{code}

The boolean values are returned from functions to indicate either
successful or erroneous execution (\verb|TRUE| or \verb|FALSE|
respectively).

%---------------------------------------
\subsubsection*{Function Return Code}
%---------------------------------------

The type \verb|ftmgs_retcode_t| defines the return code for
incremental (iterative) functions, which can be error
(\verb|FTMGS_ERROR|), success (\verb|FTMGS_OK|), or still unfinished
(\verb|FTMGS_UNFINISHED|), which means that the (iterative) operation
is still unfinished and it needs to be executed some more times,
till either error or success is found.

\begin{code}
typedef enum ftmgs_retcode_t \{
    FTMGS_ERROR, FTMGS_OK, FTMGS_UNFINISHED
\} ftmgs_retcode_t;
\end{code}

%-------------------------------------------------------------------------------
\subsection{Abstract Data Types and Function API}
%-------------------------------------------------------------------------------

The public API of FTMGS library is defined in the header file
\verb|ftmgs.h|, which should be included.

All defined data in the FTMGS library, except for enumerations, are
Abstract Data Types (ADT), and therefore their internal representation
is hidden (and protected). This fact has several advantages. 

{\footnotesize\begin{itemize}%
\item The internal representation is hidden, so there is no need to
  expose the internal data types and internal implementation.
\item As the internal representation is hidden, it diminishes the
  possibility (and temptation) to bypass the API and dealing with the
  internal data representation.
\item It improves the possibility of internal modifications that do
  not affect to the external public API.
\item It improves the binary compatibility of the library, since all
  data are defined as pointers to the internal representation, which
  can change without affecting to the external pointers.
\end{itemize}}%

As abstract data types, except for enumerations, any variable to deal
with data must be declared as a \textbf{pointer} to the hidden data
representation. Each of them will be created through a function (the
\textbf{constructor}) that allocates memory space to hold the internal
representation, and initializes the data to a known initial
state. Moreover, when such data is not useful anymore, then it is
necessary to call a function (the \textbf{destructor}) to free the
allocated resources associated with the internal
representation. Additionally, there is also a function (the
\textbf{cloner}) that allows to clone the internal representation of
the abstract data.

For each type in the library, these functions are named by the name of
the type followed by the word \emph{new}, \emph{delete} and
\emph{clone} respectively. For example, for the type \verb|rndctx_t|,
the following code defines a variable to point to the internal
representation, calls the constructors, other functions from the API,
cloning and finally calls the destructors:
\begin{code}
#include <ftmgs.h>
int main()
\{
    unsigned x;
    rndctx_t* rctx;                             /* uninitialized variable */
    rndctx_t* random_context = rndctx_t_new();  /* constructor */
    /* ... */
    bi_random_seed(random_context, PseudoEntropy);
    /* ... */
    x = bi_random_ui(10, random_context);
    /* ... */
    rctx = rndctx_t_clone(random_context);      /* cloner     */
    /* ... */
    rndctx_t_delete(random_context);            /* destructor */
    rndctx_t_delete(rctx);                      /* destructor */
    return 0;
\}
\end{code}

%-------------------------------------------------------------------------------
\subsection{Random Numbers}
%-------------------------------------------------------------------------------

The following values define the sources of entropy for seeding the
random number generator:

{\footnotesize\begin{itemize}%
\item \verb|TrueEntropy|: it seeds the random number generator with
  real random bits from a source of entropy that uses random noise
  from internal devices. If there is not enough real random bits, as
  required, from the source of entropy, then the operation blocks
  until there are enough real random bits available. This mode is
  \textbf{required} for secure cryptographic use.
\item \verb|PseudoEntropy|: it seeds the random number generator with
  random bits from a source of entropy that uses random noise from
  internal devices. If there is not enough real random bits, as
  required, from the source of entropy, then the rest of required bits
  are generated internally following some pseudo-random number
  generation. This mode is \textbf{not valid} for secure cryptographic
  use, but it can be used for other less secure requirement scenarios.
\item \verb|NoEntropy|: it does not use any source of entropy for
  seeding. It seeds the random number generator with the same fixed
  seed, so it is useful for debugging purposes, since, as it starts
  with the same seed, it repeats the same sequence of generated random
  numbers.
\end{itemize}}%
\begin{code}
enum entropy_src_t \{
    TrueEntropy, PseudoEntropy, NoEntropy
\};
\end{code}

%---------------------------------------
\subsubsection*{Abstract Data Types}
%---------------------------------------
\begin{itemize}\small
\item \verb|rndctx_t|: it is used to hold the internal context of the
  random number generator. 

  It is used extensively throughout the library, and it is required to
  have been seeded previously.
\end{itemize}
%---------------------------------------
\subsubsection*{Construction, Copy and Destruction}
%---------------------------------------
The following functions declare the constructor, cloner and destructor
for the aforementioned abstract data type:
\begin{code}
rndctx_t* rndctx_t_new();
rndctx_t* rndctx_t_clone(const rndctx_t* o);
void rndctx_t_delete(rndctx_t* p);
\end{code}
%---------------------------------------
\subsubsection*{API}
%---------------------------------------
\begin{api}
unsigned bi_random_seed(rndctx_t* rnd_ctx, unsigned entropy_src);
\end{api}
\begin{quote}\footnotesize
It allows to initialize and seed the context for the random number
generator. The source of entropy may be any of the aforementioned
values with their explained meaning. This function must be called once
before using the generator context in any other function. It returns
the amount of bytes used from the entropy source.
\end{quote}
%---------------------------------------
\begin{api}
unsigned bi_random_reseed(rndctx_t* rnd_ctx, unsigned entropy_src);
\end{api}
\begin{quote}\footnotesize
It allows to re-seed the context for the random number
generator. The source of entropy may be any of the aforementioned
values with their explained meaning. It returns
the amount of bytes used from the entropy source.
\end{quote}
%---------------------------------------
\begin{api}
void bi_random_bytes(void* buf, unsigned buflen, rndctx_t* rnd_ctx);
\end{api}
\begin{quote}\footnotesize
It generates \verb|buflen| random bytes that will be stored in the
memory pointed to by \verb|buf|, by using the previously seeded random
context \verb|rnd_ctx|. The memory pointed by \verb|buf| must have
been previously allocated with enough room to hold \verb|buflen| bytes.
\end{quote}
%---------------------------------------
\begin{api}
unsigned bi_random_ui(unsigned max, rndctx_t* rnd_ctx);
\end{api}
\begin{quote}\footnotesize
It returns a random \verb|unsigned| number between 0 and \verb|max|
(\verb|max| exclusive), by using the previously seeded random context
\verb|rnd_ctx|.
\end{quote}

%-------------------------------------------------------------------------------
\subsection{Group Creation}
%-------------------------------------------------------------------------------
The following values define the security parameters used when creating
a group:
\begin{code}
enum secpar_t \{
    Nu1 = 1024, Nu2 = 2048, Nu3 = 3072
\};
\end{code}
%---------------------------------------
\subsubsection*{Abstract Data Types}
%---------------------------------------
\begin{itemize}\small
\item \verb|ftmgs_fa_pbkey_t|: it holds the modulus \tuple{\hat{n}} and
  generator \tuple{\hat{g}} for the paillier encryption scheme.
\item \verb|ftmgs_faj_pbkey_share_t|: it holds the fairness authority
  public key share \tuple{\hat{y}_j} for the paillier encryption scheme.
\item \verb|ftmgs_faj_prkey_t|: it holds the fairness authority
  private key \tuple{\hat{o}_j} for the paillier encryption scheme. It
  allows to recover the member's tracing key.
\item \verb|ftmgs_pbkey_t|: it holds the group public key
  \tuple{n,a,a_o,b,g,h,y,\hat{n},\hat{g},\hat{y}}. It allows to verify
  group signatures, as well as provides support for all operations
  dealing with the group.
\item \verb|ftmgs_prkey_t|: it holds the group manager private key,
  the prime factors of the group modulus \tuple{p,q}. It allows to
  join new members to the group.
\item \verb|ftmgs_faj_gr_pbkey_share_t|: it holds the fairness
  authority public key share \tuple{y_j,h_j} for the el-gamal encryption
  scheme for a given group.
\item \verb|ftmgs_faj_gr_prkey_t|: it holds the fairness authority
  private key \tuple{o_j} for the el-gamal encryption scheme for a given
  group. It allows to open signatures.
\end{itemize}
%---------------------------------------
\subsubsection*{Construction, Copy and Destruction}
%---------------------------------------
The following functions declare the constructors, cloners and destructors
for the aforementioned abstract data types:
\begin{code}
ftmgs_fa_pbkey_t* ftmgs_fa_pbkey_t_new();
ftmgs_fa_pbkey_t* ftmgs_fa_pbkey_t_clone(const ftmgs_fa_pbkey_t* o);
void ftmgs_fa_pbkey_t_delete(ftmgs_fa_pbkey_t* p);

ftmgs_faj_pbkey_share_t* ftmgs_faj_pbkey_share_t_new();
tmgs_faj_pbkey_share_t* ftmgs_faj_pbkey_share_t_clone(const ftmgs_faj_pbkey_share_t* o);
void ftmgs_faj_pbkey_share_t_delete(ftmgs_faj_pbkey_share_t* p);

ftmgs_faj_prkey_t* ftmgs_faj_prkey_t_new();
ftmgs_faj_prkey_t* ftmgs_faj_prkey_t_clone(const ftmgs_faj_prkey_t* o);
void ftmgs_faj_prkey_t_delete(ftmgs_faj_prkey_t* p);

ftmgs_pbkey_t* ftmgs_pbkey_t_new();
ftmgs_pbkey_t* ftmgs_pbkey_t_clone(const ftmgs_pbkey_t* o);
void ftmgs_pbkey_t_delete(ftmgs_pbkey_t* p);

ftmgs_prkey_t* ftmgs_prkey_t_new();
ftmgs_prkey_t* ftmgs_prkey_t_clone(const ftmgs_prkey_t* o);
void ftmgs_prkey_t_delete(ftmgs_prkey_t* p);

ftmgs_faj_gr_pbkey_share_t* ftmgs_faj_gr_pbkey_share_t_new();
ftmgs_faj_gr_pbkey_share_t* ftmgs_faj_gr_pbkey_share_t_clone(const ftmgs_faj_gr_pbkey_share_t* o);
void ftmgs_faj_gr_pbkey_share_t_delete(ftmgs_faj_gr_pbkey_share_t* p);

ftmgs_faj_gr_prkey_t* ftmgs_faj_gr_prkey_t_new();
ftmgs_faj_gr_prkey_t* ftmgs_faj_gr_prkey_t_clone(const ftmgs_faj_gr_prkey_t* o);
void ftmgs_faj_gr_prkey_t_delete(ftmgs_faj_gr_prkey_t* p);
\end{code}
%---------------------------------------
\subsubsection*{API}
%---------------------------------------
The group setup process follows the sequence specified in the diagram in
section~\ref{seqdia:group}~and~\ref{seqdia:check}.

\begin{api}
void ftmgs_fa0_setup_mono(ftmgs_fa_pbkey_t* fa_pk_preimage,
                          unsigned nu,
                          rndctx_t* rnd_ctx);
\end{api}
\begin{quote}\footnotesize
It allows to create an initial public key modulus and generator for
the fairness authorities, where the security parameter \verb|nu| may
have any of the aforementioned values. It is necessary that the random
number generator context \verb|rnd_ctx| had been previously seeded.

\textbf{Note:} This function create the public key modulus and
generator for the fairness authorities for the \emph{paillier}
encryption scheme. It is necessary to be aware that the security of
the \emph{paillier} encryption scheme relies on the unknown
factorization of this public modulus, and in this case, the entity
that creates this modulus is able to know such factorization, and it
must be, therefore, a trusted authority with an overall power over the
others. Therefore, in order to improve the security of the scheme, in
next versions of this library, new functions for creating such a
public modulus in a distributed and collaborative manner will be
incorporated.

$\diamond$ In a next version of this library, a \verb|ftmgs_fa0_setup()|
function will be incorporate to implement a protocol for the
distributed generation of the RSA modulus~\cite{FS01}, to be played
among the involved fairness authorities.
\end{quote}
%---------------------------------------
\begin{api}
void ftmgs_faj_setup(ftmgs_faj_pbkey_share_t* faj_pk,
                     ftmgs_faj_pbkey_share_t* faj_pk_preimage,
                     ftmgs_faj_prkey_t* faj_sk,
                     const ftmgs_fa_pbkey_t* fa_pk,
                     rndctx_t* rnd_ctx);
\end{api}
\begin{quote}\footnotesize
It is used by each fairness authority to generate the public and
private keys (\verb|faj_pk|, and \verb|faj_sk| respectively) for a
given public modulus and generator (\verb|fa_pk|) created by using
\verb|ftmgs_fa0_setup()|. These keys are used to deal with
encryption/decryption of \emph{member tracing keys}.

Note that \verb|faj_pk_preimage| may be NULL, but otherwise it will
hold the preimage for the public key \verb|faj_pk|, that is 
\verb|faj_pk| $\equiv$ \verb|faj_pk_preimage|$^2$. This preimage is
used by the \emph{Group Manager} when creating the group.

Note that by squaring the preimage, it is assured that
$\hat{y}_j\in\mathit{QR}(n)$.
\end{quote}
%---------------------------------------
\begin{api}
void ftmgs_gm_init_setup(ftmgs_pbkey_t* gpk_preimage,
                         ftmgs_prkey_t* gsk,
                         unsigned nu,
                         rndctx_t* rnd_ctx);
\end{api}
\begin{quote}\footnotesize
It is used by the \emph{Group Manager} (GM) to compute the modulus,
generator and related terms (a preimage of the group public key
\verb|gpk|), as well as the group private key that allows the GM to
join new members to the group. The security parameter \verb|nu| may
have the aforementioned values.
\end{quote}
%---------------------------------------
\begin{api}
void ftmgs_faj_group_setup(ftmgs_faj_gr_pbkey_share_t* faj_gpk,
                           ftmgs_faj_gr_pbkey_share_t* faj_gpk_preimage,
                           ftmgs_faj_gr_prkey_t* faj_gsk,
                           const ftmgs_pbkey_t* gpk_preimage,
                           rndctx_t* rnd_ctx);
\end{api}
\begin{quote}\footnotesize
It is used by each fairness authority to generate the public and
private keys (\verb|faj_gpk|, and \verb|faj_gsk| respectively) for a
given group public key (\verb|gpk_preimage|) created by using
\verb|ftmgs_gm_init_setup()|. These keys are used to deal with
encryption/decryption for \emph{opening} signatures.

Note that \verb|faj_gpk_preimage| may be NULL, but otherwise it will
hold the preimage for the public key \verb|faj_gpk|, that is 
\verb|faj_gpk| $\equiv$ \verb|faj_gpk_preimage|$^2$. This preimage is
used by the \emph{Group Manager} when creating the group.

Note that by squaring the preimage, it is assured that
$y_j\in\mathit{QR}(n)$ and $h_j\in\mathit{QR}(n)$.
\end{quote}
%---------------------------------------
\begin{api}
ftmgs_retcode_t ftmgs_gm_group_setup(ftmgs_pbkey_t* gpk,
                                     ftmgs_pbkey_t* gpk_preimage,
                                     unsigned nfas,
                                     const ftmgs_fa_pbkey_t* fa_pk_preimage,
                                     const ftmgs_faj_pbkey_share_t* faj_pk_preimage,
                                     const ftmgs_faj_gr_pbkey_share_t* faj_gpk_preimage);
\end{api}
\begin{quote}\footnotesize
It is used by the \emph{Group Manager} (GM) to incorporate into the
group public key the public key preimages for each fairness authority
that will supervise \emph{opening} and \emph{revealing} operations fro
the group, where \verb|nfas| specifies the total number of required
fairness authorities in the process.

This function belongs to an iterative process to incorporate the
public keys for all the fairness authorities, and therefore it returns
\verb|FTMGS_UNFINISHED| while the number of incorporated fairness
authority's public keys is lower that the amount of required fairness
authorities (\verb|nfas|). When the process is over, then
either \verb|FTMGS_OK| or \verb|FTMGS_ERROR| is returned to indicate
success or failure in the operation.

Note that \verb|gpk_preimage| may be NULL, but otherwise it will
hold the preimage for the group public key \verb|gpk|, that is 
\verb|gpk| $\equiv$ \verb|gpk_preimage|$^2$. This preimage is
used by any entity to check that the group members have the right
order. Note that by squaring the preimage, it is assured that group
members belong to $\mathit{QR}(n)$.
\end{quote}
%---------------------------------------
\begin{api}
bool_t ftmgs_check_gpk(const ftmgs_pbkey_t* gpk,
                       const ftmgs_pbkey_t* gpk_preimage);
\end{api}
\begin{quote}\footnotesize
It returns \verb|TRUE| (\verb|1|) if the group public key (\verb|gpk|)
is well formed with respect to the preimage (\verb|gpk_preimage|),
which means that all members have the right order and belong to
$\mathit{QR}(n)$. It returns \verb|FALSE| (\verb|0|) otherwise.

$\diamond$ In a next version of this library, this function will also
check that the moduli in the group public key do not have small prime
factors~\cite{CM99}.
\end{quote}
%---------------------------------------
\begin{api}
ftmgs_retcode_t ftmgs_check_gpk_fa(ftmgs_pbkey_t* gpk_aux,
                                   const ftmgs_pbkey_t* gpk,
                                   const ftmgs_fa_pbkey_t* fa_pk,
                                   const ftmgs_faj_pbkey_share_t* faj_pk,
                                   const ftmgs_faj_gr_pbkey_share_t* faj_gpk);
\end{api}
\begin{quote}\footnotesize
This function belongs to an iterative process to check that the public
keys for all the fairness authorities have been incorporated to the
group public key, and therefore it returns \verb|FTMGS_UNFINISHED|
while the number of checked fairness authority's public keys is lower
that the amount of required fairness authorities previously
incorporated to the group public key. When the process is over, then
either \verb|FTMGS_OK| or \verb|FTMGS_ERROR| is returned to indicate
success or failure in the operation.

Note that the fairness authority's public keys are used, instead of
their preimages. Also note that \verb|gpk_aux| is used to temporarily
hold the incorporation of checked public keys while the iterative
process is being carried out.
\end{quote}
%---------------------------------------
\begin{api}
unsigned ftmgs_get_nfas_reveal(const ftmgs_pbkey_t* gpk);
\end{api}
\begin{quote}\footnotesize
It returns the number of fairness authorities required to reveal a
\emph{member tracing key}.
\end{quote}
%---------------------------------------
\begin{api}
unsigned ftmgs_get_nfas_open(const ftmgs_pbkey_t* gpk);
\end{api}
\begin{quote}\footnotesize
It returns the number of fairness authorities required to open a
signature.
\end{quote}
%---------------------------------------

%-------------------------------------------------------------------------------
\subsection{Joining New Members}
%-------------------------------------------------------------------------------

%---------------------------------------
\subsubsection*{Abstract Data Types}
%---------------------------------------
\begin{itemize}\small
\item \verb|dss_parms_t|: it holds the DSS parameters \tuple{p, q, g},
  used by the user when authenticated to join the group.
\item \verb|dsa_pbkey_t|: it holds the DSA user's public key
  \tuple{y}, used by the user when authenticated to join the group.
\item \verb|dsa_prkey_t|: it holds the DSA user's private key
  \tuple{x}, used by the user when authenticated to join the group.
\end{itemize}
%---------------------------------------
\begin{itemize}\small
\item \verb|dlogx_t|: it holds a user's master private key
  \tuple{x}, used by the user when authenticated to join the group. It
  can come from the user's DSA private key, or from a group member's
  private key.
\item \verb|dlog_t|: it holds a user's master public key \tuple{n, g,
  y}, used by the user when authenticated to join the group. It can
  come from the user's DSA public key and DSS parameters, or from a
  group signature issued by the member when authenticated in joining a
  new group.
\end{itemize}
%---------------------------------------
\begin{itemize}\small
\item \verb|ftmgs_join_prv_t|: it holds temporal user's private data
  generated while the iterative process of joining to a group.
\item \verb|ftmgs_join_pbl_t|: it holds temporal public data generated
  while the iterative process of joining a new member to a group.
\item \verb|ftmgs_mbr_ref_t|: it holds the member's reference
  \tuple{A_i,e_i,C_i,X_i,U_i,V_i,E^{\wp}_i,\mathcal{A}_u} (join log)
  kept by the group manager for each member of the group. It holds
  information to allow recovery, with the collaboration of the
  fairness authorities, of tracing keys and opening signatures, as
  well as non-repudiable bindings with the user.
\item \verb|ftmgs_mbr_prkey_t|: it holds the member's private key
  \tuple{A_i,e_i,x_i,x'_i} that allows the member to issue anonymous
  and unlinkable group signatures. It holds, among other data, the
  user's master key.
\end{itemize}

%---------------------------------------
\subsubsection*{Construction, Copy and Destruction}
%---------------------------------------
The following functions declare the constructors, cloners and destructors
for the aforementioned abstract data types:
\begin{code}
dss_parms_t* dss_parms_t_new();
dss_parms_t* dss_parms_t_clone(const dss_parms_t* o);
void dss_parms_t_delete(dss_parms_t* p);

dsa_pbkey_t* dsa_pbkey_t_new();
dsa_pbkey_t* dsa_pbkey_t_clone(const dsa_pbkey_t* o);
void dsa_pbkey_t_delete(dsa_pbkey_t* p);

dsa_prkey_t* dsa_prkey_t_new();
dsa_prkey_t* dsa_prkey_t_clone(const dsa_prkey_t* o);
void dsa_prkey_t_delete(dsa_prkey_t* p);
\end{code}
%---------------------------------------
\begin{code}
dlogx_t* dlogx_t_new();
dlogx_t* dlogx_t_clone(const dlogx_t* o);
void dlogx_t_delete(dlogx_t* p);

dlog_t* dlog_t_new();
dlog_t* dlog_t_clone(const dlog_t* o);
void dlog_t_delete(dlog_t* p);
\end{code}
%---------------------------------------
\begin{code}
ftmgs_join_prv_t* ftmgs_join_prv_t_new();
ftmgs_join_prv_t* ftmgs_join_prv_t_clone(const ftmgs_join_prv_t* o);
void ftmgs_join_prv_t_delete(ftmgs_join_prv_t* p);

ftmgs_join_pbl_t* ftmgs_join_pbl_t_new();
ftmgs_join_pbl_t* ftmgs_join_pbl_t_clone(const ftmgs_join_pbl_t* o);
void ftmgs_join_pbl_t_delete(ftmgs_join_pbl_t* p);

ftmgs_mbr_ref_t* ftmgs_mbr_ref_t_new();
ftmgs_mbr_ref_t* ftmgs_mbr_ref_t_clone(const ftmgs_mbr_ref_t* o);
void ftmgs_mbr_ref_t_delete(ftmgs_mbr_ref_t* p);

ftmgs_mbr_prkey_t* ftmgs_mbr_prkey_t_new();
ftmgs_mbr_prkey_t* ftmgs_mbr_prkey_t_clone(const ftmgs_mbr_prkey_t* o);
void ftmgs_mbr_prkey_t_delete(ftmgs_mbr_prkey_t* p);
\end{code}
%---------------------------------------
\subsubsection*{API}
%---------------------------------------
The join process follows the sequence specified in the diagram in
section~\ref{seqdia:join}.

\begin{api}
void extract_umk_from_dsa(dlogx_t* x, const dsa_prkey_t* dsa_sk);
\end{api}
\begin{quote}\footnotesize
It is used to extract the \emph{user's master key} from a DSA private
key that have been used (a DSA signature used as authentication
method) when the user was authorized to join the group. This user's
master key will be used in the joining process, and it will be
embedded into the user's member private key.
\end{quote}
%---------------------------------------
\begin{api}
void extract_uauth_from_dsa(dlog_t* uauth,
                            const dsa_pbkey_t* dsa_pk,
                            const dss_parms_t* dss_parms);
\end{api}
\begin{quote}\footnotesize
It is used to extract the \emph{user's authentication} from a DSA
public key that have been used (a DSA signature used as authentication
method) when the user was authorized to join the group. This user's
authentication will be used in the joining process, and it will be
embedded into the member's reference with non-repudiation purposes.
\end{quote}
%---------------------------------------
\begin{api}
void extract_umk_from_msk(dlogx_t* x, const ftmgs_mbr_prkey_t* msk);
\end{api}
\begin{quote}\footnotesize
It is used to extract the \emph{user's master key} from a FTMGS group
member's private key that have been used (a FTMGS signature used as
authentication method) when the user was authorized to join the
group. This user's master key will be used in the joining process, and
it will be embedded into the user's member private key.
\end{quote}
%---------------------------------------
\begin{api}
void extract_uauth_from_sg(dlog_t* uauth,
                           const ftmgs_sign_t* sg,
                           const ftmgs_pbkey_t* gpk);
\end{api}
\begin{quote}\footnotesize
It is used to extract the \emph{user's authentication} from a FTMGS
signature that have been used (a FTMGS signature used as
authentication method) when the user was authorized to join the
group. This user's authentication will be used in the joining process,
and it will be embedded into the member's reference with
non-repudiation purposes.
\end{quote}
%---------------------------------------
\begin{api}
void extract_dsa_from_umk(dsa_prkey_t* dsa_sk, const dlogx_t* x);
\end{api}
\begin{quote}\footnotesize
This function is used to extract (and create) a DSA private key from a
\emph{user's master key} (which can also be extracted from a
\emph{member's private key}), in this way it discourages the user from
sharing her member's private keys with other users, since in this
case, they will be able to recover her DSA private key and it will
allow them to impersonate the user in a world-wide manner.
\end{quote}
%---------------------------------------
\begin{apix}
void extract_umk_from_rsa(dlogx_t* x, const rsa_prkey_t* sk);
\end{apix}
\begin{quote}\footnotesize
It is used to extract the \emph{user's master key} from a RSA private
key that have been used (a RSA signature used as authentication
method) when the user was authorized to join the group. This user's
master key will be used in the joining process, and it will be
embedded into the user's member private key.

$\diamond$ This function is not currently defined in the API. It will
be included in a next version of this library.
\end{quote}
%---------------------------------------
\begin{apix}
void extract_uauth_from_rsa(dlog_t* uauth,
                           const rsa_sign_t* sg,
                           const rsa_pbkey_t* pk);
\end{apix}
\begin{quote}\footnotesize
It is used to extract the \emph{user's authentication} from a RSA
signature that have been used (a RSA signature used as
authentication method) when the user was authorized to join the
group. This user's authentication will be used in the joining process,
and it will be embedded into the member's reference with
non-repudiation purposes.

$\diamond$ This function is not currently defined in the API. It will
be included in a next version of this library.
\end{quote}
%---------------------------------------
\begin{apix}
void extract_rsa_from_umk(rsa_prkey_t* rsa_sk, const dlogx_t* x);
\end{apix}
\begin{quote}\footnotesize
This function is used to extract (and create) a RSA private key from a
\emph{user's master key} (which can also be extracted from a
\emph{member's private key}), in this way it discourages the user from
sharing her member's private keys with other users, since in this
case, they will be able to recover her RSA private key and it will
allow them to impersonate the user in a world-wide manner.

$\diamond$ This function is not currently defined in the API. It will
be included in a next version of this library.
\end{quote}
%---------------------------------------
\begin{api}
ftmgs_retcode_t ftmgs_join_usr(ftmgs_join_prv_t* prv,
                               ftmgs_join_pbl_t* pbl,
                               ftmgs_mbr_prkey_t* msk,
                               const ftmgs_pbkey_t* gpk,
                               const dlogx_t* umk,
                               const dlog_t* u_auth,
                               rndctx_t* rnd_ctx);
\end{api}
\begin{quote}\footnotesize
This function belongs to an iterative protocol, at the user's side, to
join a new member to a FTMGS group, and therefore it returns
\verb|FTMGS_UNFINISHED| while the protocol is still unfinished. When
the protocol is over, then either \verb|FTMGS_OK| or
\verb|FTMGS_ERROR| is returned to indicate success or failure in the
operation.

The protocol follows the sequence specified in the diagram in
section~\ref{seqdia:join}. The protocol starts at the user's side,
then the private and public outcome (\verb|prv| and \verb|pbl|) are
stored for the next iteration, and the public outcome (\verb|pbl|) is
sent (usually its ASN.1 DER encoding) to the group manager, which
plays its part or the protocol, then the public outcome (\verb|pbl|)
is stored for the next iteration, and sent (usually its ASN.1 DER
encoding) again to the user, which plays again this protocol till the
end. Note that the user's side of the protocol is the starting and
ending points of the protocol.

As result, if everything was fine, the user gets her \emph{member's
  private key} (\verb|msk|) that allows her to issue FTMGS group
signatures, and therefore, being authenticated as a member of the
group.

If the user was authorized (and authenticated) to join the group by
any external means, then the \emph{user's master key} (\verb|umk|) and
\emph{user's authentication} (\verb|u_auth|) should be extracted from
the external authentication, otherwise they should be
\verb|NULL|. Note that both, user and group manager, should follow the
same conventions for these cases in order to play a valid join protocol.

Note that \verb|prv| and \verb|pbl| are used to temporarily hold
respectively the private and public data between iterations while the
iterative protocol is being carried out.
\end{quote}
%---------------------------------------
\begin{api}
ftmgs_retcode_t ftmgs_join_gm(ftmgs_join_pbl_t* pbl,
                              ftmgs_mbr_ref_t* mr,
                              const ftmgs_pbkey_t* gpk,
                              const ftmgs_prkey_t* gsk,
                              const dlog_t* u_auth,
                              rndctx_t* rnd_ctx);
\end{api}
\begin{quote}\footnotesize
This function belongs to an iterative protocol, at the group manager's
side, to join a new member to a FTMGS group, and therefore it returns
\verb|FTMGS_UNFINISHED| while the protocol is still unfinished. When
the protocol is over, then either \verb|FTMGS_OK| or
\verb|FTMGS_ERROR| is returned to indicate success or failure in the
operation.

The protocol follows the sequence specified in the diagram in
section~\ref{seqdia:join}. The protocol starts at the user's side,
then the private and public outcome (\verb|prv| and \verb|pbl|) are
stored for the next iteration, and the public outcome (\verb|pbl|) is
sent (usually its ASN.1 DER encoding) to the group manager, which
plays its part or the protocol, then the public outcome (\verb|pbl|)
is stored for the next iteration, and sent (usually its ASN.1 DER
encoding) again to the user, which plays again this protocol till the
end. Note that the user's side of the protocol is the starting and
ending points of the protocol.

As result, if everything was fine, the group manager gets the \emph{member's
  reference} (\verb|mr|) that allows the Judge (with collaboration of
GM and FAs) opening and reveal operations.

If the user was authorized (and authenticated) to join the group by
any external means, then the \emph{user's authentication}
(\verb|u_auth|) should be extracted from the external authentication,
otherwise it should be \verb|NULL|. Note that both, user and group
manager, should follow the same conventions for these cases in order
to play a valid join protocol.

Note that \verb|pbl| is used to temporarily hold the public data
between iterations while the iterative protocol is being carried out.
\end{quote}
%---------------------------------------
\begin{api}
bool_t ftmgs_vrfy_join_log(const ftmgs_mbr_ref_t* mr,
                           const ftmgs_pbkey_t* gpk);
\end{api}
\begin{quote}\footnotesize
It returns \verb|TRUE| (\verb|1|) if the member's reference
(\verb|mr|) is consistent and well formed, which means that the
non-repudiable bindings and proofs still hold, and therefore the
record has not been manipulated. It returns \verb|FALSE| (\verb|0|)
otherwise.
\end{quote} 
%---------------------------------------

%-------------------------------------------------------------------------------
\subsection{Signing and Verifying}
%-------------------------------------------------------------------------------

The following values allow to choose the operation mode for the SHA
hash engine:
\begin{code}
enum sha_mode_t \{
    Sha1, Sha224, Sha256, Sha384, Sha512
\};
\end{code}

%---------------------------------------
\subsubsection*{Abstract Data Types}
%---------------------------------------
\begin{itemize}\small
\item \verb|ftmgs_sign_t|: it holds a FTMGS signature
  \tuple{T_1,\cdots,T_7,\sigma^{\wp}} issued by a member of the group.
\end{itemize}
%---------------------------------------
\subsubsection*{Construction, Copy and Destruction}
%---------------------------------------
The following functions declare the constructor, cloner and destructor
for the aforementioned abstract data type:
\begin{code}
ftmgs_sign_t* ftmgs_sign_t_new();
ftmgs_sign_t* ftmgs_sign_t_clone(const ftmgs_sign_t* o);
void ftmgs_sign_t_delete(ftmgs_sign_t* p);
\end{code}
%---------------------------------------
\subsubsection*{API}
%---------------------------------------
The signing process follows the sequence specified in the diagram in
section~\ref{seqdia:sign}.

\begin{api}
bool_t ftmgs_sign_dat(unsigned which_sha,
                     ftmgs_sign_t* sg,
                     const void* dat, unsigned datlen,
                     const ftmgs_pbkey_t* gpk,
                     const ftmgs_mbr_prkey_t* msk,
                     rndctx_t* rnd_ctx);
\end{api}
\begin{quote}\footnotesize
It issues a FTMGS signature (\verb|sg|) for group public key
(\verb|gpk|) with member's private key (\verb|msk|). The SHA engine,
as selected by \verb|which_sha| from aforementioned values, is applied
to some given data bytes (\verb|dat|) of length (\verb|datlen|).

It returns \verb|TRUE| (\verb|1|) if everything was fine, and returns
\verb|FALSE| (\verb|0|) otherwise.

Compatibility note: when issuing the signature, the \emph{digest} is
calculated by applying the selected SHA engine to the user's data, and
then the signature it is issued over this digest as specified in
the next functions:\\
{$\mathit{digest} = \mathrm{SHA}_w(\underbrace{\mathit{dat}}_{\mathit{datlen}})$}
\end{quote}
%---------------------------------------
\begin{api}
bool_t ftmgs_vrfy_dat(unsigned which_sha,
                      const ftmgs_sign_t* sg,
                      const void* dat, unsigned datlen,
                      const ftmgs_pbkey_t* gpk);
\end{api}
\begin{quote}\footnotesize
It returns \verb|TRUE| (\verb|1|) if a FTMGS signature (\verb|sg|) can
be verified for group public key (\verb|gpk|), and returns
\verb|FALSE| (\verb|0|) otherwise. The SHA engine, as selected by
\verb|which_sha| from aforementioned values, is applied to some given
data bytes (\verb|dat|) of length (\verb|datlen|).

Compatibility note: when verifying a signature, the \emph{digest} is
calculated by applying the selected SHA engine to the user's data, and
then the verification it is applied over this digest as specified in
the next functions:\\
{$\mathit{digest} = \mathrm{SHA}_w(\underbrace{\mathit{dat}}_{\mathit{datlen}})$}
\end{quote}
%---------------------------------------
\begin{api}
bool_t ftmgs_sign_dgst(ftmgs_sign_t* sg,
                      const void* dat_digest,
                      unsigned dat_digestlen,
                      const ftmgs_pbkey_t* gpk,
                      const ftmgs_mbr_prkey_t* msk,
                      rndctx_t* rnd_ctx);
\end{api}
\begin{quote}\footnotesize
It issues a FTMGS signature (\verb|sg|) for group public key
(\verb|gpk|) with member's private key (\verb|msk|). The signature is
applied to some given data digest (\verb|dat_digest|) of length
(\verb|dat_digestlen|) that has been generated by some hashing
functions~\ref{hashing}.

It returns \verb|TRUE| (\verb|1|) if everything was fine, and returns
\verb|FALSE| (\verb|0|) otherwise.

Compatibility note: when issuing the signature, the SHA hash
(truncated to security parameter $k$ bits) is applied to the following
data in the same order as specified (numbers are represented as a
\emph{big-endian} byte sequence):\\ {$c =
  \mathrm{SHA}_k(\mathit{digest}||B_1||\cdots||B_6||\underbrace{n||\cdots||n}_6||g||h||T_2^{-1}||T_5||T_7||y||T_1^{-1}||a||b||a_0||T_3||T_4||T_6)$}
\end{quote}
%---------------------------------------
\begin{api}
bool_t ftmgs_vrfy_dgst(const ftmgs_sign_t* sg,
                       const void* dat_digest,
                       unsigned dat_digestlen,
                       const ftmgs_pbkey_t* gpk);
\end{api}
\begin{quote}\footnotesize
It returns \verb|TRUE| (\verb|1|) if a FTMGS signature (\verb|sg|) can
be verified for group public key (\verb|gpk|) , and returns
\verb|FALSE| (\verb|0|) otherwise. The verification is applied to some
given data digest (\verb|dat_digest|) of length (\verb|dat_digestlen|)
that has been generated by some hashing functions~\ref{hashing}.

Compatibility note: when verifying a signature, the SHA hash
(truncated to security parameter $k$ bits) is applied to the following
data in the same order as specified (numbers are represented as a
\emph{big-endian} byte sequence):\\ {$c' =
  \mathrm{SHA}_k(\mathit{digest}||B_1||\cdots||B_6||\underbrace{n||\cdots||n}_6||g||h||T_2^{-1}||T_5||T_7||y||T_1^{-1}||a||b||a_0||T_3||T_4||T_6)$}
\end{quote}
%-------------------------------------------------------------------------------
\subsection{Signature Opening}
%-------------------------------------------------------------------------------

%---------------------------------------
\subsubsection*{Abstract Data Types}
%---------------------------------------
\begin{itemize}\small
\item \verb|ftmgs_opensharej_t|: it holds a decryption share
  \tuple{\hat{\omega}_{j\sigma},\hat{\omega}^{\wp}_{j\sigma}} of the
  opening of a signature.
\item \verb|ftmgs_openacc_t|: it holds the incremental product of the
  opening decryption shares.
\item \verb|ftmgs_open_t|: it holds the member's reference
  \tuple{\omega_{\sigma}} result of opening a signature.
\end{itemize}
%---------------------------------------
\subsubsection*{Construction, Copy and Destruction}
%---------------------------------------
The following functions declare the constructors, cloners and destructors
for the aforementioned abstract data types:
\begin{code}
ftmgs_opensharej_t* ftmgs_opensharej_t_new();
ftmgs_opensharej_t* ftmgs_opensharej_t_clone(const ftmgs_opensharej_t* o);
void ftmgs_opensharej_t_delete(ftmgs_opensharej_t* p);

ftmgs_openacc_t* ftmgs_openacc_t_new();
ftmgs_openacc_t* ftmgs_openacc_t_clone(const ftmgs_openacc_t* o);
void ftmgs_openacc_t_delete(ftmgs_openacc_t* p);

ftmgs_open_t* ftmgs_open_t_new();
ftmgs_open_t* ftmgs_open_t_clone(const ftmgs_open_t* o);
void ftmgs_open_t_delete(ftmgs_open_t* p);
\end{code}
%---------------------------------------
\subsubsection*{API}
%---------------------------------------
The opening process follows the sequence specified in the diagram in
section~\ref{seqdia:open}.

\begin{api}
bool_t ftmgs_open_dshare_j(ftmgs_opensharej_t* osj,
                         const ftmgs_sign_t* sg,
                         const ftmgs_faj_gr_pbkey_share_t* faj_gpk,
                         const ftmgs_faj_gr_prkey_t* faj_gsk,
                         const ftmgs_pbkey_t* gpk,
                         rndctx_t* rnd_ctx);
\end{api}
\begin{quote}\footnotesize
It is used by a fairness authority (with key-pair \verb|faj_gsk| and
\verb|faj_gpk|) to generate a decryption share (\verb|osj|) of the
opening of a FTMGS signature (\verb|sg|) for group public key
\verb|gpk|.

It returns \verb|TRUE| (\verb|1|) if everything was fine, and returns
\verb|FALSE| (\verb|0|) otherwise.

Compatibility note: when issuing the proof of correctness, the SHA hash
(truncated to security parameter $k$ bits) is applied to the following
data in the same order as specified (numbers are represented as a
\emph{big-endian} byte sequence):\\ {$c =
  \mathrm{SHA}_k(\mathit{digest}||B_1||B_2||n||n||g||y_j||T_2||\hat{\omega}_{j\sigma})$}
\\ where 
$\mathit{digest} = \mathrm{SHA}_k(T_1||\cdots||T_7||c||s_x||s_{x'}||s_e||s_r||s_{h'})$
\end{quote}
%---------------------------------------
\begin{api}
ftmgs_retcode_t ftmgs_open_sign(ftmgs_open_t* op,
                                ftmgs_openacc_t* osa,
                                const ftmgs_sign_t* sg,
                                const ftmgs_opensharej_t* osj,
                                const ftmgs_faj_gr_pbkey_share_t* faj_gpk,
                                const ftmgs_pbkey_t* gpk);
\end{api}
\begin{quote}\footnotesize
This function belongs to an iterative process to incrementally open a
FTMGS signature (\verb|sg|) for all the fairness authority's opening
decryption shares (\verb|osj|), and therefore it returns
\verb|FTMGS_UNFINISHED| while the number of added decryption shares is
lower that the amount of required fairness authorities. When the
process is over, then either \verb|FTMGS_OK| or \verb|FTMGS_ERROR| is
returned to indicate success or failure in the operation.

In case of success, \verb|op| holds the outcome of the opening
operation, and \verb|osa| is a temporary value that must be held while
the iterative process is not finished. Note that if the decryption
share is not properly constructed with respect to the fairness
authority's public key (\verb|faj_gpk|) and group public key
(\verb|gpk|), then the operation will fail.
\end{quote}
%---------------------------------------
\begin{api}
bool_t ftmgs_open_check(const ftmgs_open_t* op,
                        const ftmgs_mbr_ref_t* mr,
                        const ftmgs_pbkey_t* gpk);
\end{api}
\begin{quote}\footnotesize
It checks if the outcome of the opening of a signature (\verb|op|)
matches a given member's reference (\verb|mr|), and in such case, it
also checks that this member's reference is consistent
(\verb|ftmgs_vrfy_join_log()|).

It returns \verb|TRUE| (\verb|1|) if both match, and returns
\verb|FALSE| (\verb|0|) otherwise.
\end{quote}
%---------------------------------------

%-------------------------------------------------------------------------------
\subsection{Member Tracing}
%-------------------------------------------------------------------------------

%---------------------------------------
\subsubsection*{Abstract Data Types}
%---------------------------------------
\begin{itemize}\small
\item \verb|ftmgs_mtkey_sharej_t|: it holds a decryption share
  \tuple{\hat{\tau}_{ji},\hat{\tau}^{\wp}_{ji}} of the
  revealing of a member's tracing key.
\item \verb|ftmgs_mtkey_acc_t|: it holds the incremental product of the
  revealing decryption shares.
\item \verb|ftmgs_mtkey_t|: it holds the member's tracing key
  \tuple{\tau_{i}} result of opening a signature.
\end{itemize}
%---------------------------------------
\subsubsection*{Construction, Copy and Destruction}
%---------------------------------------
The following functions declare the constructors, cloners and destructors
for the aforementioned abstract data types:
\begin{code}
ftmgs_mtkey_sharej_t* ftmgs_mtkey_sharej_t_new();
ftmgs_mtkey_sharej_t* ftmgs_mtkey_sharej_t_clone(const ftmgs_mtkey_sharej_t* o);
void ftmgs_mtkey_sharej_t_delete(ftmgs_mtkey_sharej_t* p);

ftmgs_mtkey_acc_t* ftmgs_mtkey_acc_t_new();
ftmgs_mtkey_acc_t* ftmgs_mtkey_acc_t_clone(const ftmgs_mtkey_acc_t* o);
void ftmgs_mtkey_acc_t_delete(ftmgs_mtkey_acc_t* p);

ftmgs_mtkey_t* ftmgs_mtkey_t_new();
ftmgs_mtkey_t* ftmgs_mtkey_t_clone(const ftmgs_mtkey_t* o);
void ftmgs_mtkey_t_delete(ftmgs_mtkey_t* p);
\end{code}
%---------------------------------------
\subsubsection*{API}
%---------------------------------------
The reveal process follows the sequence specified in the diagram in
section~\ref{seqdia:reveal}.

\begin{api}
bool_t ftmgs_reveal_dshare_j(ftmgs_mtkey_sharej_t* mtk_shj,
                             const ftmgs_mbr_ref_t* mr,
                             const ftmgs_faj_pbkey_share_t* faj_pk,
                             const ftmgs_faj_prkey_t* faj_sk,
                             const ftmgs_pbkey_t* gpk,
                             rndctx_t* rnd_ctx);
\end{api}
\begin{quote}\footnotesize
It is used by a fairness authority (with key-pair \verb|faj_sk| and
\verb|faj_pk|) to generate a decryption share (\verb|mtk_shj|) of the
revealing the member's tracing key of a member's reference (\verb|mr|)
for group public key \verb|gpk|. It also checks that this member's
reference is consistent (\verb|ftmgs_vrfy_join_log()|).

It returns \verb|TRUE| (\verb|1|) if everything was fine, and returns
\verb|FALSE| (\verb|0|) otherwise.

Compatibility note: when issuing the proof of correctness, the SHA hash
(truncated to security parameter $k$ bits) is applied to the following
data in the same order as specified (numbers are represented as a
\emph{big-endian} byte sequence):\\ {$c =
  \mathrm{SHA}_k(\mathit{digest}||B_1||B_2||\hat{n}^2||\hat{n}^2||\hat{g}||\hat{y}_j||U_i||\hat{\tau}_{ji})$}
\\ where 
$\mathit{digest} = \mathrm{SHA}_k(A_i||e_i||C_i||X_i||U_i||V_i||g||y||n||c||s_{x'}||s_r||s_x)$
\end{quote}
%---------------------------------------
\begin{api}
ftmgs_retcode_t ftmgs_reveal_mtkey(ftmgs_mtkey_t* mtk,
                                   ftmgs_mtkey_acc_t* mtka,
                                   const ftmgs_mtkey_sharej_t* mtk_shj,
                                   const ftmgs_mbr_ref_t* mr,
                                   const ftmgs_faj_pbkey_share_t* faj_pk,
                                   const ftmgs_pbkey_t* gpk);
\end{api}
\begin{quote}\footnotesize
This function belongs to an iterative process to incrementally reveal
a member's tracing key for a given member's reference (\verb|mr|) for
all the fairness authority's revealing decryption shares (\verb|mtk_shj|),
and therefore it returns \verb|FTMGS_UNFINISHED| while the number of
added decryption shares is lower that the amount of required fairness
authorities. When the process is over, then either \verb|FTMGS_OK| or
\verb|FTMGS_ERROR| is returned to indicate success or failure in the
operation.

In case of success, \verb|mtk| holds the outcome of the reveal
operation, and \verb|mtka| is a temporary value that must be held while
the iterative process is not finished. Note that if the decryption
share is not properly constructed with respect to the fairness
authority's public key (\verb|faj_pk|) and group public key
(\verb|gpk|), then the operation will fail.
\end{quote}
%---------------------------------------
\begin{api}
bool_t ftmgs_trace(const ftmgs_sign_t* sg, const ftmgs_mtkey_t* mtk);
\end{api}
\begin{quote}\footnotesize
It checks if the outcome of revealing a member's tracing key (\verb|mtk|)
matches a given group signature (\verb|sg|).

It returns \verb|TRUE| (\verb|1|) if both match, and returns
\verb|FALSE| (\verb|0|) otherwise.
\end{quote}
%---------------------------------------

%-------------------------------------------------------------------------------
\subsection{Claiming Authorship}
%-------------------------------------------------------------------------------

%---------------------------------------
\subsubsection*{Abstract Data Types}
%---------------------------------------
\begin{itemize}\small
\item \verb|ftmgs_claim_t|: it holds the proof \tuple{\pi^{\wp}} of
  the claiming of a signature authorship.
\end{itemize}
%---------------------------------------
\subsubsection*{Construction, Copy and Destruction}
%---------------------------------------
The following functions declare the constructor, cloner and destructor
for the aforementioned abstract data type:
\begin{code}
ftmgs_claim_t* ftmgs_claim_t_new();
ftmgs_claim_t* ftmgs_claim_t_clone(const ftmgs_claim_t* o);
void ftmgs_claim_t_delete(ftmgs_claim_t* p);
\end{code}
%---------------------------------------
\subsubsection*{API}
%---------------------------------------
\begin{api}
bool_t ftmgs_claim_dat(unsigned which_sha,
                       ftmgs_claim_t* clm,
                       const ftmgs_sign_t* sg,
                       const void* dat,
                       unsigned datlen,
                       const ftmgs_pbkey_t* gpk,
                       const ftmgs_mbr_prkey_t* msk,
                       rndctx_t* rnd_ctx);
\end{api}
\begin{quote}\footnotesize
It issues a FTMGS claim (\verb|clm|) for group public key (\verb|gpk|)
with member's private key (\verb|msk|) for a signature
(\verb|sg|). The SHA engine, as selected by \verb|which_sha| from
aforementioned values, is applied to some given data bytes
(\verb|dat|) of length (\verb|datlen|).

It returns \verb|TRUE| (\verb|1|) if everything was fine, and returns
\verb|FALSE| (\verb|0|) otherwise.

Compatibility note: when issuing the claim, the \emph{digest} is
calculated by applying the selected SHA engine to the user's data, and
then the claim it is issued over this digest as specified in
the next functions:\\
{$\mathit{digest} = \mathrm{SHA}_w(\underbrace{\mathit{dat}}_{\mathit{datlen}})$}
\end{quote}
%---------------------------------------
\begin{api}
bool_t ftmgs_vrfy_claim_dat(unsigned which_sha,
                            const ftmgs_claim_t* clm,
                            const ftmgs_sign_t* sg,
                            const void* dat,
                            unsigned datlen,
                            const ftmgs_pbkey_t* gpk);
\end{api}
\begin{quote}\footnotesize
It returns \verb|TRUE| (\verb|1|) if a FTMGS claim (\verb|clm|) can be
verified for signature (\verb|sg|) and group public key (\verb|gpk|),
and returns \verb|FALSE| (\verb|0|) otherwise. The SHA engine, as
selected by \verb|which_sha| from aforementioned values, is applied to
some given data bytes (\verb|dat|) of length (\verb|datlen|).

Compatibility note: when verifying a claim, the \emph{digest} is
calculated by applying the selected SHA engine to the user's data, and
then the verification it is applied over this digest as specified in
the next functions:\\
{$\mathit{digest} = \mathrm{SHA}_w(\underbrace{\mathit{dat}}_{\mathit{datlen}})$}
\end{quote}
%---------------------------------------
\begin{api}
bool_t ftmgs_claim_dgst(ftmgs_claim_t* clm,
                        const ftmgs_sign_t* sg,
                        const void* dat_digest,
                        unsigned dat_digestlen,
                        const ftmgs_pbkey_t* gpk,
                        const ftmgs_mbr_prkey_t* msk,
                        rndctx_t* rnd_ctx);
\end{api}
\begin{quote}\footnotesize
It issues a FTMGS claim (\verb|clm|) for group public key (\verb|gpk|)
with member's private key (\verb|msk|) for a signature (\verb|sg|).
The claim is applied to some given data digest (\verb|dat_digest|) of
length (\verb|dat_digestlen|) that has been generated by some hashing
functions~\ref{hashing}.

It returns \verb|TRUE| (\verb|1|) if everything was fine, and returns
\verb|FALSE| (\verb|0|) otherwise.

Compatibility note: when issuing the claim, the SHA hash
(truncated to security parameter $k$ bits) is applied to the following
data in the same order as specified (numbers are represented as a
\emph{big-endian} byte sequence):\\ {$c =
  \mathrm{SHA}_k(\mathit{dgst}||B_1||n||T_7||T_6)$}
\\ where 
$\mathit{dgst} = \mathrm{SHA}_k(\mathit{digest}||T_1||\cdots||T_7||c||s_x||s_{x'}||s_e||s_r||s_{h'})$
\end{quote}
%---------------------------------------
\begin{api}
bool_t ftmgs_vrfy_claim_dgst(const ftmgs_claim_t* clm,
                             const ftmgs_sign_t* sg,
                             const void* dat_digest,
                             unsigned dat_digestlen,
                             const ftmgs_pbkey_t* gpk);
\end{api}
\begin{quote}\footnotesize
It returns \verb|TRUE| (\verb|1|) if a FTMGS claim (\verb|clm|) can be
verified for signature (\verb|sg|) and group public key (\verb|gpk|),
and returns \verb|FALSE| (\verb|0|) otherwise.  The verification is
applied to some given data digest (\verb|dat_digest|) of length
(\verb|dat_digestlen|) that has been generated by some hashing
functions~\ref{hashing}.

Compatibility note: when verifying the claim, the SHA hash
(truncated to security parameter $k$ bits) is applied to the following
data in the same order as specified (numbers are represented as a
\emph{big-endian} byte sequence):\\ {$c =
  \mathrm{SHA}_k(\mathit{dgst}||B_1||n||T_7||T_6)$}
\\ where 
$\mathit{dgst} = \mathrm{SHA}_k(\mathit{digest}||T_1||\cdots||T_7||c||s_x||s_{x'}||s_e||s_r||s_{h'})$
\end{quote}

%-------------------------------------------------------------------------------
\subsection{Linking Signatures}
%-------------------------------------------------------------------------------

%---------------------------------------
\subsubsection*{Abstract Data Types}
%---------------------------------------
\begin{itemize}\small
\item \verb|ftmgs_link_t|: it holds the proof \tuple{\lambda^{\wp}} of
  the linking of several FTMGS signatures.
\end{itemize}
%---------------------------------------
\subsubsection*{Construction, Copy and Destruction}
%---------------------------------------
The following functions declare the constructor, cloner and destructor
for the aforementioned abstract data type:
\begin{code}
ftmgs_link_t* ftmgs_link_t_new();
ftmgs_link_t* ftmgs_link_t_clone(const ftmgs_link_t* o);
void ftmgs_link_t_delete(ftmgs_link_t* p);
\end{code}
%---------------------------------------
\subsubsection*{API}
%---------------------------------------
\begin{api}
bool_t ftmgs_link_dat(unsigned which_sha,
                      ftmgs_link_t* lnk,
                      const void* dat, unsigned datlen,
                      const ftmgs_sign_t* sg0,
                      const ftmgs_pbkey_t* gpk0,
                      const ftmgs_sign_t* sg1,
                      const ftmgs_pbkey_t* gpk1,
                      const ftmgs_mbr_prkey_t* msk,
                      rndctx_t* rnd_ctx);
\end{api}
\begin{quote}\footnotesize
It issues a FTMGS link (\verb|lnk|) with member's private key
(\verb|msk|) for two signatures (\verb|sg0|, \verb|sg1|) for group public keys
(\verb|gpk0|, \verb|gpk1|). The SHA engine, as selected by
\verb|which_sha| from aforementioned values, is applied to some given
data bytes (\verb|dat|) of length (\verb|datlen|).

It returns \verb|TRUE| (\verb|1|) if everything was fine, and returns
\verb|FALSE| (\verb|0|) otherwise.

Compatibility note: when issuing the link, the \emph{digest} is
calculated by applying the selected SHA engine to the user's data, and
then the link it is issued over this digest as specified in
the next functions:\\
{$\mathit{digest} = \mathrm{SHA}_w(\underbrace{\mathit{dat}}_{\mathit{datlen}})$}
\end{quote}
%---------------------------------------
\begin{api}
bool_t ftmgs_vrfy_link_dat(unsigned which_sha,
                           const ftmgs_link_t* lnk,
                           const void* dat, unsigned datlen,
                           const ftmgs_sign_t* sg0,
                           const ftmgs_pbkey_t* gpk0,
                           const ftmgs_sign_t* sg1,
                           const ftmgs_pbkey_t* gpk1);
\end{api}
\begin{quote}\footnotesize
It returns \verb|TRUE| (\verb|1|) if a FTMGS link (\verb|lnk|) can be
verified for signatures (\verb|sg0|, \verb|sg1|) and group public keys
(\verb|gpk0|, \verb|gpk1|), and returns \verb|FALSE| (\verb|0|)
otherwise. The SHA engine, as selected by \verb|which_sha| from
aforementioned values, is applied to some given data bytes
(\verb|dat|) of length (\verb|datlen|).

Compatibility note: when verifying a link, the \emph{digest} is
calculated by applying the selected SHA engine to the user's data, and
then the verification it is applied over this digest as specified in
the next functions:\\
{$\mathit{digest} = \mathrm{SHA}_w(\underbrace{\mathit{dat}}_{\mathit{datlen}})$}
\end{quote}
%---------------------------------------
\begin{api}
bool_t ftmgs_link_m_dat(unsigned which_sha,
                        ftmgs_link_t* lnk,
                        const void* dat,
                        unsigned datlen,
                        unsigned nsg,
                        const ftmgs_sign_t* sg[],
                        const ftmgs_pbkey_t* gpk[],
                        const ftmgs_mbr_prkey_t* msk,
                        rndctx_t* rnd_ctx);
\end{api}
\begin{quote}\footnotesize
It is the same as previous \verb|ftmgs_link_dat()|, but applied to an
array of signatures.
\end{quote}
%---------------------------------------
\begin{api}
bool_t ftmgs_vrfy_link_m_dat(unsigned which_sha,
                             const ftmgs_link_t* lnk,
                             const void* dat,
                             unsigned datlen,
                             unsigned nsg,
                             const ftmgs_sign_t* sg[],
                             const ftmgs_pbkey_t* gpk[]);
\end{api}
\begin{quote}\footnotesize
It is the same as previous \verb|ftmgs_vrfy_link_dat()|, but applied to an
array of signatures.
\end{quote}
%---------------------------------------
\begin{api}
bool_t ftmgs_link_dgst(ftmgs_link_t* lnk,
                       const void* dat_digest,
                       unsigned dat_digestlen,
                       const ftmgs_sign_t* sg0,
                       const ftmgs_pbkey_t* gpk0,
                       const ftmgs_sign_t* sg1,
                       const ftmgs_pbkey_t* gpk1,
                       const ftmgs_mbr_prkey_t* msk,
                       rndctx_t* rnd_ctx);
\end{api}
\begin{quote}\footnotesize
It issues a FTMGS link (\verb|lnk|) with member's private key
(\verb|msk|) for two signatures (\verb|sg0|, \verb|sg1|) for group
public keys (\verb|gpk0|, \verb|gpk1|). The link is applied to some
given data digest (\verb|dat_digest|) of length (\verb|dat_digestlen|)
that has been generated by some hashing functions~\ref{hashing}.

It returns \verb|TRUE| (\verb|1|) if everything was fine, and returns
\verb|FALSE| (\verb|0|) otherwise.

Compatibility note: when issuing the link, the SHA hash
(truncated to security parameter $k$ bits) is applied to the following
data in the same order as specified (numbers are represented as a
\emph{big-endian} byte sequence):\\ {$c =
  \mathrm{SHA}_k(\mathit{dgst}||B_1||B_2||n_1||n_2||T_{17}||T_{16}||T_{27}||T_{26})$}
\\ where 
$\mathit{dgst} = \mathrm{SHA}_k(\mathit{digest}||T_{11}||\cdots||T_{17}||c_1||s_{1x}||s_{1x'}||s_{1e}||s_{1r}||s_{1h'}||T_{21}||\cdots||T_{27}||c_2||s_{2x}||s_{2x'}||s_{2e}||s_{2r}||s_{2h'})$
\end{quote}
%---------------------------------------
\begin{api}
bool_t ftmgs_vrfy_link_dgst(const ftmgs_link_t* lnk,
                            const void* dat_digest,
                            unsigned dat_digestlen,
                            const ftmgs_sign_t* sg0,
                            const ftmgs_pbkey_t* gpk0,
                            const ftmgs_sign_t* sg1,
                            const ftmgs_pbkey_t* gpk1);
\end{api}
\begin{quote}\footnotesize
It returns \verb|TRUE| (\verb|1|) if a FTMGS link (\verb|lnk|) can be
verified for signatures (\verb|sg0|, \verb|sg1|) and group public keys
(\verb|gpk0|, \verb|gpk1|), and returns \verb|FALSE| (\verb|0|)
otherwise. The verification is applied to some given data digest
(\verb|dat_digest|) of length (\verb|dat_digestlen|) that has been
generated by some hashing functions~\ref{hashing}.

Compatibility note: when verifying the link, the SHA hash
(truncated to security parameter $k$ bits) is applied to the following
data in the same order as specified (numbers are represented as a
\emph{big-endian} byte sequence):\\ {$c =
  \mathrm{SHA}_k(\mathit{dgst}||B_1||B_2||n_1||n_2||T_{17}||T_{16}||T_{27}||T_{26})$}
\\ where 
$\mathit{dgst} = \mathrm{SHA}_k(\mathit{digest}||T_{11}||\cdots||T_{17}||c_1||s_{1x}||s_{1x'}||s_{1e}||s_{1r}||s_{1h'}||T_{21}||\cdots||T_{27}||c_2||s_{2x}||s_{2x'}||s_{2e}||s_{2r}||s_{2h'})$
\end{quote}
%---------------------------------------
\begin{api}
bool_t ftmgs_link_m_dgst(ftmgs_link_t* lnk,
                         const void* dat_digest,
                         unsigned dat_digestlen,
                         unsigned nsg,
                         const ftmgs_sign_t* sg[],
                         const ftmgs_pbkey_t* gpk[],
                         const ftmgs_mbr_prkey_t* msk,
                         rndctx_t* rnd_ctx);
\end{api}
\begin{quote}\footnotesize
It is the same as previous \verb|ftmgs_link_m_dgst()|, but applied to an
array of signatures.
\end{quote}
%---------------------------------------
\begin{api}
bool_t ftmgs_vrfy_link_m_dgst(const ftmgs_link_t* lnk,
                              const void* dat_digest,
                              unsigned dat_digestlen,
                              unsigned nsg,
                              const ftmgs_sign_t* sg[],
                              const ftmgs_pbkey_t* gpk[]);
\end{api}
\begin{quote}\footnotesize
It is the same as previous \verb|ftmgs_vrfy_link_m_dgst()|, but applied to an
array of signatures.
\end{quote}

%-------------------------------------------------------------------------------
\subsection{Hashing}\label{hashing}
%-------------------------------------------------------------------------------

The hash API may be used to create the digest of some data that is not
directly available as an array of contiguous data bytes.

The following values allow to choose the operation mode for the SHA
hash engine, as well as the length (in bytes) of the hash digest:
\begin{code}
enum sha_mode_t \{
    Sha1, Sha224, Sha256, Sha384, Sha512
\};
enum sha_size_t \{
    Sha1Size = 20, Sha224Size = 28, Sha256Size = 32,
    Sha384Size = 48, Sha512Size = 64, ShaMaxSize = Sha512Size
\};
\end{code}
%---------------------------------------
\subsubsection*{Abstract Data Types}
%---------------------------------------
\begin{itemize}\small
\item \verb|shactx_t|: it is used to hold the internal context of the
  SHA engine.
\end{itemize}
%---------------------------------------
\subsubsection*{Construction, Copy and Destruction}
%---------------------------------------
The following functions declare the constructor, cloner and destructor
for the aforementioned abstract data type:
\begin{code}
shactx_t* shactx_t_new();
shactx_t* shactx_t_clone(const shactx_t* o);
void shactx_t_delete(shactx_t* p);
\end{code}
%---------------------------------------
\subsubsection*{API}
%---------------------------------------
\begin{api}
bool_t sha_reset(shactx_t* sha_ctx, unsigned sha_mode);
\end{api}
\begin{quote}\footnotesize
It resets the context for the SHA engine, as selected by
\verb|sha_mode| from the aforementioned values.
\end{quote}
%---------------------------------------
\begin{api}
bool_t sha_input(shactx_t* sha_ctx, const void* dat, unsigned datlen);
\end{api}
\begin{quote}\footnotesize
It incrementally incorporates a sequence of \verb|datlen| bytes stored
in \verb|dat| to the SHA context.
\end{quote}
%---------------------------------------
\begin{api}
bool_t sha_result(shactx_t* sha_ctx, void* dat_digest, unsigned* digestlen);
\end{api}
\begin{quote}\footnotesize
It generates a \emph{digest} of the data previously incorporated into
the SHA context. This digest is stored into \verb|dat_digest|, which
should have enough allocated memory to hold the result, which is
specified by \verb|digestlen| (a maximum value requests the length
specified for the previously selected SHA engine, see aforementioned
values for sha sizes). As result, \verb|digestlen| also holds the
length of the generated digest.
\end{quote}

%-------------------------------------------------------------------------------
\subsection{Data Buffer}
%-------------------------------------------------------------------------------

The data buffer is used to hold a buffer of bytes representing the
ASN.1 DER encoding of some abstract data type.

%---------------------------------------
\subsubsection*{Abstract Data Types}
%---------------------------------------
\begin{itemize}\small
\item \verb|buffer_t|: it holds a buffer of bytes dynamically
  allocated in the heap. It holds \emph{data} in a memory with a
  \emph{maxsize} that can be expanded or shrinked as required, as well
  as the \emph{size} of the data currently stored in the buffer.
\end{itemize}
%---------------------------------------
\subsubsection*{Construction, Copy and Destruction}
%---------------------------------------
The following functions declare the constructor, cloner and destructor
for the aforementioned abstract data type:
\begin{code}
buffer_t* buffer_t_new();
buffer_t* buffer_t_clone(const buffer_t* o);
void buffer_t_delete(buffer_t* p);
\end{code}
%---------------------------------------
\subsubsection*{API}
%---------------------------------------
\begin{api}
char* buffer_data(const buffer_t* buff);
\end{api}
\begin{quote}\footnotesize
It returns a pointer to the data bytes stored in the buffer.
\end{quote}
%---------------------------------------
\begin{api}
unsigned buffer_size(const buffer_t* buff);
\end{api}
\begin{quote}\footnotesize
It returns the size of the data bytes stored in the buffer.
\end{quote}
%---------------------------------------
\begin{api}
unsigned buffer_maxsize(const buffer_t* buff);
\end{api}
\begin{quote}\footnotesize
It returns the size of the memory already allocated in the buffer to
hold the data bytes.
\end{quote}
%---------------------------------------
\begin{api}
void buffer_reserve(buffer_t* buff, unsigned maxsz);
\end{api}
\begin{quote}\footnotesize
If \verb|maxsz > buffer_maxsize(buff)|, then it increments the size of the
memory already allocated in the buffer to the value of \verb|maxsz|.
\end{quote}
%---------------------------------------
\begin{api}
void buffer_shrink(buffer_t* buff);
\end{api}
\begin{quote}\footnotesize
If \verb|maxsz > buffer_size(buff)+1|, then it shrinks the size of the
memory already allocated in the buffer to the value of \verb|buffer_size(buff)+1|.
\end{quote}
%---------------------------------------
\begin{api}
void buffer_reset(buffer_t* buff);
\end{api}
\begin{quote}\footnotesize
It resets the size and maxsize of the buffer to zero, as well as
deallocates the reserved memory.
\end{quote}
%---------------------------------------
\begin{api}
void buffer_clear(buffer_t* buff);
\end{api}
\begin{quote}\footnotesize
It resets the size and maxsize of the buffer to zero.
\end{quote}
%---------------------------------------
\begin{api}
void buffer_resize(buffer_t* buff, unsigned newsz);
\end{api}
\begin{quote}\footnotesize
It increments the size of the data bytes already stored in the buffer.
\end{quote}
%---------------------------------------
\begin{api}
void buffer_push(buffer_t* buff, const void* dat, unsigned datlen);
\end{api}
\begin{quote}\footnotesize
It appends \verb|datlen| bytes of data pointed to by \verb|dat| to the
currently stored data in the buffer.
\end{quote}
%---------------------------------------
\begin{api}
void buffer_push_str(buffer_t* buff, const char* str);
\end{api}
\begin{quote}\footnotesize
It appends the bytes of a null terminated string of chars pointed to
by \verb|str| to the currently stored data in the buffer, as well as
it also puts the null termination char (\verb|'\0'|), but it is not
included in the count of stored data.
\end{quote}
%---------------------------------------
\begin{api}
void buffer_push_buf(buffer_t* buff, const buffer_t* b);  /* MACRO */
\end{api}
\begin{quote}\footnotesize
\begin{code}
buffer_push(buff, buffer_data(b), buffer_size(b));
\end{code}
\end{quote}
%---------------------------------------
\begin{api}
void buffer_resize_add(buffer_t* buff, unsigned sz);      /* MACRO */
\end{api}
\begin{quote}\footnotesize
\begin{code}
buffer_resize(buff, buffer_size(buffer)+sz);
\end{code}
\end{quote}
%---------------------------------------
\begin{api}
void buffer_reserve_add(buffer_t* buff, unsigned sz);     /* MACRO */
\end{api}
\begin{quote}\footnotesize
\begin{code}
buffer_reserve(buff, buffer_size(buffer)+sz+1);
\end{code}
\end{quote}
%---------------------------------------

%-------------------------------------------------------------------------------
\subsection{ASN.1 Conversion of Data Structures}
%-------------------------------------------------------------------------------
The following values define the return codes from the ASN.1 conversion functions:
\begin{code}
typedef enum asn1_retcode_t \{
    ASN1_Success,
    ASN1_File_Not_Found,
    ASN1_Element_Not_Found,
    ASN1_Identifier_Not_Found,
    ASN1_Der_Error,
    ASN1_Value_Not_Found,
    ASN1_Generic_Error,
    ASN1_Value_Not_Valid,
    ASN1_Tag_Error,
    ASN1_Tag_Implicit,
    ASN1_Error_Type_Any,
    ASN1_Syntax_Error,
    ASN1_Mem_Error,
    ASN1_Mem_Alloc_Error,
    ASN1_Der_Overflow,
    ASN1_Name_Too_Long,
    ASN1_Array_Error,
    ASN1_Element_Not_Empty
\} asn1_retcode_t;
\end{code}
%---------------------------------------
\subsubsection*{API}
%---------------------------------------
The following functions allow to encode in ASN.1 DER each of the
aforementioned abstract data types, storing the result in the
aforementioned \verb|buffer_t| buffer of bytes. They also allow to
decode from a buffer of bytes in ASN.1 DER to any of the
aforementioned abstract data types:
\begin{code}
asn1_retcode_t asn1_enc_dssparms(buffer_t* buff, const dss_parms_t* p);
asn1_retcode_t asn1_dec_dssparms(dss_parms_t* p, const void* buff, int len);

asn1_retcode_t asn1_enc_dsapbkey(buffer_t* buff, const dsa_pbkey_t* p);
asn1_retcode_t asn1_dec_dsapbkey(dsa_pbkey_t* p, const void* buff, int len);

asn1_retcode_t asn1_enc_dsaprkey(buffer_t* buff, const dsa_prkey_t* p);
asn1_retcode_t asn1_dec_dsaprkey(dsa_prkey_t* p, const void* buff, int len);

asn1_retcode_t asn1_enc_fapbkey(buffer_t* buff, const ftmgs_fa_pbkey_t* p);
asn1_retcode_t asn1_dec_fapbkey(ftmgs_fa_pbkey_t* p, const void* buff, int len);

asn1_retcode_t asn1_enc_grpbkey(buffer_t* buff, const ftmgs_pbkey_t* p);
asn1_retcode_t asn1_dec_grpbkey(ftmgs_pbkey_t* p, const void* buff, int len);

asn1_retcode_t asn1_enc_grprkey(buffer_t* buff, const ftmgs_prkey_t* p);
asn1_retcode_t asn1_dec_grprkey(ftmgs_prkey_t* p, const void* buff, int len);

asn1_retcode_t asn1_enc_fapbkeysh(buffer_t* buff, const ftmgs_faj_pbkey_share_t* p);
asn1_retcode_t asn1_dec_fapbkeysh(ftmgs_faj_pbkey_share_t* p, const void* buff, int len);

asn1_retcode_t asn1_enc_faprkey(buffer_t* buff, const ftmgs_faj_prkey_t* p);
asn1_retcode_t asn1_dec_faprkey(ftmgs_faj_prkey_t* p, const void* buff, int len);

asn1_retcode_t asn1_enc_fagrpbkeysh(buffer_t* buff, const ftmgs_faj_gr_pbkey_share_t* p);
asn1_retcode_t asn1_dec_fagrpbkeysh(ftmgs_faj_gr_pbkey_share_t* p, const void* buff, int len);

asn1_retcode_t asn1_enc_fagrprkey(buffer_t* buff, const ftmgs_faj_gr_prkey_t* p);
asn1_retcode_t asn1_dec_fagrprkey(ftmgs_faj_gr_prkey_t* p, const void* buff, int len);

asn1_retcode_t asn1_enc_dlog(buffer_t* buff, const dlog_t* p);
asn1_retcode_t asn1_dec_dlog(dlog_t* p, const void* buff, int len);

asn1_retcode_t asn1_enc_dlogx(buffer_t* buff, const dlogx_t* p);
asn1_retcode_t asn1_dec_dlogx(dlogx_t* p, const void* buff, int len);

asn1_retcode_t asn1_enc_joinpbl(buffer_t* buff, const ftmgs_join_pbl_t* p);
asn1_retcode_t asn1_dec_joinpbl(ftmgs_join_pbl_t* p, const void* buff, int len);

asn1_retcode_t asn1_enc_joinprv(buffer_t* buff, const ftmgs_join_prv_t* p);
asn1_retcode_t asn1_dec_joinprv(ftmgs_join_prv_t* p, const void* buff, int len);

asn1_retcode_t asn1_enc_mbrref(buffer_t* buff, const ftmgs_mbr_ref_t* p);
asn1_retcode_t asn1_dec_mbrref(ftmgs_mbr_ref_t* p, const void* buff, int len);

asn1_retcode_t asn1_enc_mbrprkey(buffer_t* buff, const ftmgs_mbr_prkey_t* p);
asn1_retcode_t asn1_dec_mbrprkey(ftmgs_mbr_prkey_t* p, const void* buff, int len);

asn1_retcode_t asn1_enc_sign(buffer_t* buff, const ftmgs_sign_t* p);
asn1_retcode_t asn1_dec_sign(ftmgs_sign_t* p, const void* buff, int len);

asn1_retcode_t asn1_enc_openshare(buffer_t* buff, const ftmgs_opensharej_t* p);
asn1_retcode_t asn1_dec_openshare(ftmgs_opensharej_t* p, const void* buff, int len);

asn1_retcode_t asn1_enc_openacc(buffer_t* buff, const ftmgs_openacc_t* p);
asn1_retcode_t asn1_dec_openacc(ftmgs_openacc_t* p, const void* buff, int len);

asn1_retcode_t asn1_enc_open(buffer_t* buff, const ftmgs_open_t* p);
asn1_retcode_t asn1_dec_open(ftmgs_open_t* p, const void* buff, int len);

asn1_retcode_t asn1_enc_mtkeyshare(buffer_t* buff, const ftmgs_mtkey_sharej_t* p);
asn1_retcode_t asn1_dec_mtkeyshare(ftmgs_mtkey_sharej_t* p, const void* buff, int len);

asn1_retcode_t asn1_enc_mtkeyacc(buffer_t* buff, const ftmgs_mtkey_acc_t* p);
asn1_retcode_t asn1_dec_mtkeyacc(ftmgs_mtkey_acc_t* p, const void* buff, int len);

asn1_retcode_t asn1_enc_mtkey(buffer_t* buff, const ftmgs_mtkey_t* p);
asn1_retcode_t asn1_dec_mtkey(ftmgs_mtkey_t* p, const void* buff, int len);

asn1_retcode_t asn1_enc_claim(buffer_t* buff, const ftmgs_claim_t* p);
asn1_retcode_t asn1_dec_claim(ftmgs_claim_t* p, const void* buff, int len);

asn1_retcode_t asn1_enc_link(buffer_t* buff, const ftmgs_link_t* p);
asn1_retcode_t asn1_dec_link(ftmgs_link_t* p, const void* buff, int len);
\end{code}
%-------------------------------------------------------------------------------
%-------------------------------------------------------------------------------
%-------------------------------------------------------------------------------
\section{ASN.1 Definition of FTMGS Data Structures}
%-------------------------------------------------------------------------------
%-------------------------------------------------------------------------------
\subsection{Size of ASN.1 DER Encoding of FTMGS Data Structures}
%-------------------------------------------------------------------------------
\newcommand{\tpsz}[2]{\begin{tabular}{c}\texttt{#1}\\#2\end{tabular}}

Size (in bytes) of the ASN.1 DER encoding of FTMGS data structures for
the following security parameters [Nu: 1024, K: 128].

\subsubsection*{Group Setup}
{\noindent\footnotesize
\tpsz{ftmgs\_fa\_pbkey\_t}{412}
\tpsz{ftmgs\_pbkey\_t}{1607}
\tpsz{ftmgs\_prkey\_t}{137}
\tpsz{ftmgs\_faj\_pbkey\_share\_t}{264}
\tpsz{ftmgs\_faj\_prkey\_t}{264}
\tpsz{ftmgs\_faj\_gr\_pbkey\_share\_t}{266}
\tpsz{ftmgs\_faj\_gr\_prkey\_t}{68}
}

\subsubsection*{Join New Member (without User's Authentication)}
{\noindent\footnotesize
\tpsz{ftmgs\_join\_prv\_t:u1}{352}
\tpsz{ftmgs\_join\_pbl\_t:u1}{751}
\tpsz{ftmgs\_join\_pbl\_t:gm2}{54}
\tpsz{ftmgs\_join\_prv\_t:u3}{207}
\tpsz{ftmgs\_join\_pbl\_t:u3}{1326}
\tpsz{ftmgs\_join\_pbl\_t:gm4}{240}
\tpsz{ftmgs\_mbr\_ref\_t}{1282}
\tpsz{ftmgs\_mbr\_prkey\_t}{336}
}

\subsubsection*{Join New Member (with User's DSA Authentication)}
{\noindent\footnotesize
\tpsz{dss\_parms\_t}{290}
\tpsz{dsa\_pbkey\_t}{132}
\tpsz{dsa\_prkey\_t}{22}
\tpsz{dlogx\_t}{24}
\tpsz{dlog\_t}{399}
}

{\noindent\footnotesize
\tpsz{ftmgs\_join\_prv\_t:u1}{340}
\tpsz{ftmgs\_join\_pbl\_t:u1}{750}
\tpsz{ftmgs\_join\_pbl\_t:gm2}{54}
\tpsz{ftmgs\_join\_prv\_t:u3}{207}
\tpsz{ftmgs\_join\_pbl\_t:u3}{1345}
\tpsz{ftmgs\_join\_pbl\_t:gm4}{240}
\tpsz{ftmgs\_mbr\_ref\_t}{1688}
\tpsz{ftmgs\_mbr\_prkey\_t}{324}
}

\subsubsection*{Sign / Verify}
{\noindent\footnotesize
\tpsz{ftmgs\_sign\_t}{1308}
\tpsz{ftmgs\_sign\_t}{(with user's auth)\\1326}
}

\subsubsection*{Open / Check}
{\noindent\footnotesize
\tpsz{ftmgs\_opensharej\_t}{218}
\tpsz{ftmgs\_openacc\_t}{137}
\tpsz{ftmgs\_open\_t}{134}
}

\subsubsection*{Reveal / Trace}
{\noindent\footnotesize
\tpsz{ftmgs\_mtkey\_sharej\_t}{542}
\tpsz{ftmgs\_mtkey\_acc\_t}{267}
\tpsz{ftmgs\_mtkey\_t}{201}
}

\subsubsection*{Claim}
{\noindent\footnotesize
\tpsz{ftmgs\_claim\_t}{55}
\tpsz{ftmgs\_claim\_t}{(with user's auth)\\73}
}

\subsubsection*{Link}
{\noindent\footnotesize
\tpsz{ftmgs\_link\_t}{55}
\tpsz{ftmgs\_link\_t}{(with user's auth)\\73}
}

%-------------------------------------------------------------------------------
%-------------------------------------------------------------------------------
%-------------------------------------------------------------------------------
\subsection{ASN.1 Definition of FTMGS Data Structures}
%-------------------------------------------------------------------------------
\begin{quote}\footnotesize
\begin{verbatim}
Ftmgs { 1 2 3 4 }   -- DUMMY OID
DEFINITIONS IMPLICIT TAGS ::=
BEGIN

-- DSA
id-dsa-base OBJECT IDENTIFIER ::= {
    iso(1) member-body(2) us(840) x9-57(10040) x9cm(4) 
}

id-dsa OBJECT IDENTIFIER ::= {
    id-dsa-base 1 
}

id-dsa-with-sha1 OBJECT IDENTIFIER ::=  {
    id-dsa-base 3
}

-- RFC-3279
Dss-Parms ::= SEQUENCE  {
    p       INTEGER,
    q       INTEGER,
    g       INTEGER  
}

-- RFC-3279
DSAPublicKey ::= INTEGER -- public key, Y 

-- RFC-5958 pg.5
DSAPrivateKey ::= INTEGER -- private key, x

-- RFC-3279
Dss-Sig-Value ::=  SEQUENCE  {
    r       INTEGER,
    s       INTEGER  
}

-- Object Identifiers
id-ftmgs-base OBJECT IDENTIFIER ::= {
    1 2 3 4   -- some arcs to be registered DUMMY OID
}

id-ftmgs OBJECT IDENTIFIER ::= {
    id-ftmgs-base 1 
}

id-ftmgs-with-sha1 OBJECT IDENTIFIER ::=  {
    id-ftmgs-base 3 
}

id-ftmgs-with-sha256 OBJECT IDENTIFIER ::=  {
    id-ftmgs-base 5 
}

id-ftmgs-with-sha512 OBJECT IDENTIFIER ::=  {
    id-ftmgs-base 7 
}

-- Unsigned Small Integer (16 bits)
UInt16 ::= INTEGER (0..65535)

-- Syspar Structure Definition
SysPar ::= SEQUENCE {
    nu  UInt16,
    k   UInt16
}

-- Fairness Authorities' Public Key
FAPbKey ::= SEQUENCE {
    sp      SysPar,
    nkeys   UInt16,
    n       INTEGER,
    g       INTEGER,
    y       INTEGER
}

-- Fairness Authorities' Group Public Key
FAGrPbKey ::= SEQUENCE {
    sp      SysPar,
    nkeys   UInt16,
    n       INTEGER,
    g       INTEGER,
    y       INTEGER
}

-- Group Public Key
GrPbKey ::= SEQUENCE {
    gmpk    FAGrPbKey,
    a0      INTEGER,
    a       INTEGER,
    b       INTEGER,
    h       INTEGER,
    fapk    FAPbKey
}

-- Group Manager's Private Key [JOIN]
GrPrKey ::= SEQUENCE {
    p   INTEGER,
    q   INTEGER
}

-- Fairness Authority's Public Key
FAPbKeyShare ::= SEQUENCE {
    yj      INTEGER
}

-- Fairness Authority's Private Key [REVEAL]
FAPrKey ::= SEQUENCE {
    xj      INTEGER
}

-- Fairness Authority's Group Public Key
FAGrPbKeyShare ::= SEQUENCE {
    yj      INTEGER,
    hj      INTEGER
}

-- Fairness Authority's Group Private Key [OPEN]
FAGrPrKey ::= SEQUENCE {
    xj      INTEGER
}

-- User's Master Public Key (from DSA or FTMGS-Signature)
DLog ::= SEQUENCE {
    n           INTEGER,
    g           INTEGER,
    y           INTEGER
}

-- User's Master Private Key (from DSA o FTMGS-Signature)
DLogX ::= SEQUENCE {
    x           INTEGER
}

-- Join Proof
JoinProof ::= SEQUENCE {
    c           INTEGER,
    sx1         INTEGER,
    sr          INTEGER,
    sx          INTEGER
}

JoinU1Prv ::= SEQUENCE {
    nadrp-xx    INTEGER,
    nadrp-rr    INTEGER,
    x1i         INTEGER
}

JoinU1Pbl ::= SEQUENCE {
    nadrp-C1    INTEGER,
    nadrp-C2    INTEGER,
    nadrp-c     INTEGER,
    nadrp-sx    INTEGER,
    nadrp-sr    INTEGER,
    ci          INTEGER
}

JoinGM2Pbl ::= SEQUENCE {
    nadrp-yy    INTEGER
}

JoinU3Prv ::= SEQUENCE {
    nadrp-x     INTEGER,
    xi          INTEGER
}

JoinU3Pbl ::= SEQUENCE {
    nadrp-C3    INTEGER,
    nadrp-c     INTEGER,
    nadrp-sx    INTEGER,
    nadrp-sz    INTEGER,
    nadrp-sr    INTEGER,
    ui          INTEGER,
    vi          INTEGER,
    eiproof     JoinProof
}

JoinGM4Pbl ::= SEQUENCE {
    ai          INTEGER,
    ei          INTEGER
}

JoinPbl ::= SEQUENCE {
    status  UInt16,
    data CHOICE {
        u1      [0] JoinU1Pbl,
        gm2     [1] JoinGM2Pbl,
        u3      [2] JoinU3Pbl,
        gm4     [3] JoinGM4Pbl,
        error   [4] NULL
    }
}

JoinPrv ::= SEQUENCE {
    status  UInt16,
    data CHOICE {
        u1      [0] JoinU1Prv,
        u3      [1] JoinU3Prv,
        error   [2] NULL
    }
}

-- Member's Reference
MbrRef ::= SEQUENCE {
    ai          INTEGER,
    ei          INTEGER,
    ci          INTEGER,
    xi          INTEGER,
    ui          INTEGER,
    vi          INTEGER,
    uauth   [0] DLog        OPTIONAL,
    eiproof     JoinProof
}

-- Member's Private Key
MbrPrKey ::= SEQUENCE {
    ai          INTEGER,
    ei          INTEGER,
    xi          INTEGER,
    x1i         INTEGER
}

-- Sign
Sign ::= SEQUENCE {
    t1          INTEGER,
    t2          INTEGER,
    t3          INTEGER,
    t4          INTEGER,
    t5          INTEGER,
    t6          INTEGER,
    t7          INTEGER,
    c           INTEGER,
    sx          INTEGER,
    sx1         INTEGER,
    se          INTEGER,
    sr          INTEGER,
    sh1         INTEGER
}

-- Open
Open ::= SEQUENCE {
    a           INTEGER
}

OpenShare ::= SEQUENCE {
    alphax      INTEGER,
    c           INTEGER,
    sx          INTEGER
}

OpenAcc ::= SEQUENCE {
    nshares     UInt16,
    a           INTEGER
}

-- Reveal&Trace
MTKey ::= SEQUENCE {
    n           INTEGER,
    x           INTEGER
}

MTKeyShare ::= SEQUENCE {
    alphax      INTEGER,
    c           INTEGER,
    sx          INTEGER
}

MTKeyAcc ::= SEQUENCE {
    nshares     UInt16,
    a           INTEGER
}

-- Claim
Claim ::= SEQUENCE {
    c           INTEGER,
    sx          INTEGER
}

-- Link
Link ::= SEQUENCE {
    c           INTEGER,
    sx          INTEGER
}

END
\end{verbatim}
\end{quote}
%-------------------------------------------------------------------------------
%-------------------------------------------------------------------------------
%-------------------------------------------------------------------------------
\section{Usage Example}
%-------------------------------------------------------------------------------
The attached file \verb|test.c|, licensed under GNU GPLv2, is an
example of using the FTMGS library public API.
%-------------------------------------------------------------------------------
%-------------------------------------------------------------------------------
%-------------------------------------------------------------------------------
\begin{thebibliography}{10}

\bibitem{bcly:2008:ftmgs}
V.~Benjumea, S.~G. Choi, J.~Lopez, and M.~Yung.
\newblock {F}air {T}raceable {M}ulti-{G}roup {S}ignatures.
\newblock In G.~Tsudik, editor, {\em {FC}'08: 12th. Intl. Conf. on Financial
  Cryptography and Data Security}, volume 5143 of {\em Lecture Notes in
  Computer Science}, pages 231--246. Springer-Verlag, Jan. 2008.
  Full Version: \texttt{http://eprint.iacr.org/2008/047}.

\bibitem{CM99}
J.~Camenisch and M.~Michels.
\newblock  Proving in zero-knowledge that a number is the product of
    two safe primes. 
\newblock In {\em EUROCRYPT'99}, vol. 1592 of {LNCS}, pp. 107--122,
    Springer-Verlag, 1999 

\bibitem{CS00}
R.~Cramer and V.~Shoup.
\newblock {S}ignature {S}chemes {B}ased on the {S}trong {RSA} {A}ssumption.
\newblock In {\em 6th ACM Conf on Computer and Communication Security}, 1999
  {\em ACM Transactions on Information and System Security}, May 9, 2000.

\bibitem{KTY04}
A.~Kiayias, Y.~Tsiounis, and M.~Yung.
\newblock {T}raceable {S}ignatures.
\newblock In {\em EUROCRYPT'04}, pages 571--589, 2004.
\newblock Full Version: \texttt{http://eprint.iacr.org/2004/007}.

\bibitem{FS01}
P.~Fouque and J.~Stern.
\newblock Fully distributed threshold {RSA} under standard assumptions.
\newblock In {\em ASIACRYPT}, 2001.

\end{thebibliography}
%-------------------------------------------------------------------------------
%-------------------------------------------------------------------------------
%-------------------------------------------------------------------------------
%-------------------------------------------------------------------------------
\end{document}
%-------------------------------------------------------------------------------
