%-------------------------------------------------------------------------------
% Copyright (C)  2012  Vicente Benjumea, University of Malaga, Spain
% 
% Redistribution and use in source (LaTeX) and 'compiled' forms (SGML,
% HTML, PDF, PostScript, RTF and so forth) with or without
% modification, are permitted provided that the following conditions
% are met:
% 
%  1. Redistributions of source code (LaTeX) must retain the
%     above copyright notice, this list of conditions and the following
%     disclaimer as the first lines of this file unmodified.
% 
%  2. Redistributions in compiled form (transformed to other DTDs,
%    converted to PDF, PostScript, RTF and other formats) must
%    reproduce the above copyright notice, this list of conditions and
%    the following disclaimer in the documentation and/or other
%    materials provided with the distribution.
% 
% THIS DOCUMENTATION IS PROVIDED BY THE COPYRIGHT HOLDERS "AS IS" AND
% ANY EXPRESS OR IMPLIED WARRANTIES, INCLUDING, BUT NOT LIMITED TO,
% THE IMPLIED WARRANTIES OF MERCHANTABILITY AND FITNESS FOR A
% PARTICULAR PURPOSE ARE DISCLAIMED. IN NO EVENT SHALL THE COPYRIGHT
% HOLDERS BE LIABLE FOR ANY DIRECT, INDIRECT, INCIDENTAL, SPECIAL,
% EXEMPLARY, OR CONSEQUENTIAL DAMAGES (INCLUDING, BUT NOT LIMITED TO,
% PROCUREMENT OF SUBSTITUTE GOODS OR SERVICES; LOSS OF USE, DATA, OR
% PROFITS; OR BUSINESS INTERRUPTION) HOWEVER CAUSED AND ON ANY THEORY
% OF LIABILITY, WHETHER IN CONTRACT, STRICT LIABILITY, OR TORT
% (INCLUDING NEGLIGENCE OR OTHERWISE) ARISING IN ANY WAY OUT OF THE
% USE OF THIS DOCUMENTATION, EVEN IF ADVISED OF THE POSSIBILITY OF
% SUCH DAMAGE.
%-------------------------------------------------------------------------------
\documentclass[a4paper]{article}
%%\usepackage{latexsym}
%%\usepackage{amssymb}
%%\usepackage{amsfonts}
%%\usepackage{multicol}
\usepackage{alltt}
%-------------------------------------------------------------------------------
\newenvironment{code}%
{\begin{quote}\footnotesize\begin{alltt}}%
{\end{alltt}\end{quote}}%
%---------------------------------------
\newenvironment{api}%
{\noindent$\bullet$\hfill\begin{minipage}[t]{0.97\linewidth}\footnotesize\begin{alltt}}%
{\end{alltt}\end{minipage}}%
%---------------------------------------
\newenvironment{apix}%
{\noindent$\diamond$\hfill\begin{minipage}[t]{0.97\linewidth}\footnotesize\begin{alltt}}%
{\end{alltt}\end{minipage}}%
%-------------------------------------------------------------------------------
\newcommand{\tuple}[1]{\ensuremath{\langle #1 \rangle}}
%-------------------------------------------------------------------------------
\title{Development Workflow Using GIT}
\author{Vicente Benjumea}
\date{(DRAFT)}
\bibliographystyle{abbrv}
%-------------------------------------------------------------------------------
\newcommand{\text}[2]{\makebox[#1]{\raisebox{-.65ex}{#2}}}
\newcommand{\version}[1]{\makebox[8.5ex]{\oval(25,10)\raisebox{-.65ex}{\makebox[0pt][c]{#1}}}}%
\newcommand{\commit}[1]{\makebox[5ex]{\oval(15,10)\raisebox{-.65ex}{\makebox[0pt][c]{#1}}}}%
\newcommand{\merge}[1]{\makebox[3.5ex]{\oval(10,10)\raisebox{-.65ex}{\makebox[0pt][c]{#1}}}}%
\newcommand{\arrow}{\makebox[3.5ex]{\vector(1,0){10}}}
\newcommand{\arrowup}{\makebox[0pt][l]{\vector(1,2){10}}}
\newcommand{\arrowdown}{\makebox[0pt][l]{\vector(1,-2){10}}}
\newcommand{\arrowdup}{\makebox[0pt][l]{\vector(1,4){10}}}
\newcommand{\arrowddown}{\makebox[0pt][l]{\vector(1,-4){10}}}
\newcommand{\arrowlen}[1]{\makebox[#1]{\rule[0pt]{#1}{0.4pt}\vector(1,0){0}}}
%-------------------------------------------------------------------------------
\begin{document}
\maketitle
%-------------------------------------------------------------------------------
\subsection*{Abstract}
%-------------------------------------------------------------------------------
\begin{quote}\small
This document describes the workflow for development using
GIT. It is largely based on the \verb|gitworkflows| manual
page\footnote{\texttt{http://schacon.github.com/git/gitworkflows.html}}. So
many rules presented in this guide, as well as their rationale, can be
found in such docu-ment.
\end{quote}
%-------------------------------------------------------------------------------
\section{Version Numbering}
%-------------------------------------------------------------------------------
Version numbering of development uses three numbers: \emph{Version
  Major}, \emph{Version Minor} and \emph{Revision Number}, with the
following meanings: \par{\footnotesize\begin{itemize}%
\item \emph{Version Major}: each number in the sequence defines a new
  release that incorporates major changes in source code that affect
  the \emph{public interface}, API and functionality. It usually
  incorporates new added features, and the public interface may not be
  compatible with a previous release.
\item \emph{Version Minor}: each number in the sequence defines a new
  release that incorporates changes in source code that do not affect
  the \emph{public} behavior of the interface. The incorporated
  changes usually improve the way things are done, but do not add new
  features. The public interface should be fully compatible with
  previous release.
\item \emph{Revision Number}: each number in the sequence defines a
  new release that incorporates changes in source code that fixes some
  bugs present in previous releases and do not affect the
  \emph{public} behavior of the interface.
\end{itemize}}%
%-------------------------------------------------------------------------------
%\newpage
\vspace*{\fill}\noindent
\begin{minipage}{\linewidth}\scriptsize
\hrulefill\\
\noindent
Copyright (c)  2012  Vicente Benjumea, University of Malaga, Spain\\

\noindent
Redistribution and use in source (LaTeX) and 'compiled' forms
(SGML, HTML, PDF, PostScript, RTF and so forth) with or without
modification, are permitted provided that the following conditions are
met:
\begin{enumerate}
\item Redistributions of source code (LaTeX) must retain the
      above copyright notice, this list of conditions and the
      following disclaimer as the first lines of this file unmodified.

\item Redistributions in compiled form (transformed to other DTDs,
      converted to PDF, PostScript, RTF and other formats) must
      reproduce the above copyright notice, this list of conditions
      and the following disclaimer in the documentation and/or other
      materials provided with the distribution.
\end{enumerate}
THIS DOCUMENTATION IS PROVIDED BY THE COPYRIGHT HOLDERS "AS IS" AND
ANY EXPRESS OR IMPLIED WARRANTIES, INCLUDING, BUT NOT LIMITED TO, THE
IMPLIED WARRANTIES OF MERCHANTABILITY AND FITNESS FOR A PARTICULAR
PURPOSE ARE DISCLAIMED. IN NO EVENT SHALL THE COPYRIGHT HOLDERS BE
LIABLE FOR ANY DIRECT, INDIRECT, INCIDENTAL, SPECIAL, EXEMPLARY, OR
CONSEQUENTIAL DAMAGES (INCLUDING, BUT NOT LIMITED TO, PROCUREMENT OF
SUBSTITUTE GOODS OR SERVICES; LOSS OF USE, DATA, OR PROFITS; OR
BUSINESS INTERRUPTION) HOWEVER CAUSED AND ON ANY THEORY OF LIABILITY,
WHETHER IN CONTRACT, STRICT LIABILITY, OR TORT (INCLUDING NEGLIGENCE
OR OTHERWISE) ARISING IN ANY WAY OUT OF THE USE OF THIS DOCUMENTATION,
EVEN IF ADVISED OF THE POSSIBILITY OF SUCH DAMAGE.
\end{minipage}
%-------------------------------------------------------------------------------
\newpage
%-------------------------------------------------------------------------------
\section{Development Branches}
%-------------------------------------------------------------------------------

There are three main development \emph{branches} (\emph{maint},
\emph{master} and \emph{unstable}), as well as some maintenance
branches for previous releases, and a \emph{topic} branch for each 
bug-fix, minor changes and new added features.

\begin{itemize}%
\item The \emph{maint} branch tracks the commits for bug-fixes for the
  last released version, and hold all released revisions for the
  previous released version. For example, if the last released version
  is \verb|0.1.0|, then this branch tracks all later bug-fix commits
  and revisions, such as \verb|0.1.1|, \verb|0.1.2|, \verb|0.1.3|, and
  so on.
\item The \emph{master} branch tracks the commits for minor and major
  changes that should go into the next released version
  (\verb|X.Y.0|).
\item The \emph{unstable} branch is used to commit minor and major
  changes for testing their stability. After successful testing,
  such changes can be incorporated into the \emph{master} branch.
\begin{center}
\fbox{\begin{picture}(230,80)\sffamily\scriptsize%
\put(0,0){Time:$\,\longmapsto$}
\put(0,60){\text{10ex}{Unstable:}\phantom{\version{v0.1.0}\arrow}\commit{c8}\arrow\commit{c10}}
\put(0,40){\text{10ex}{Master:}\version{v0.1.0}\arrowup\arrowdown\arrow\commit{c6}\arrow\commit{c7}\arrow\commit{c9}\arrow\version{v0.2.0}}
\put(0,20){\text{10ex}{Maint:}\phantom{\version{v0.1.0}\arrow}\commit{c1}\arrow\commit{c3}\arrow\version{v0.1.1}\arrow\commit{c5}\arrow\version{v0.1.2}}
\end{picture}}%
\end{center}

\item As new versions are relased, older maintenance branches are
  renamed after the version number they maintain, e.g.:
  \emph{maint-0.1}
\begin{center}
\fbox{\begin{picture}(270,100)\sffamily\scriptsize%
\put(0,0){Time:$\,\longmapsto$}
\put(0,80){\text{10ex}{Unstable:}\phantom{\version{v0.1.0}\arrowup\arrowddown\arrow\commit{c6}\arrow\commit{c7}\arrow\commit{c9}\arrow\version{v0.2.0}\arrow}\commit{c13}}
\put(0,60){\text{10ex}{Master:}\version{v0.1.0}\arrowddown\arrow\commit{c6}\arrow\commit{c7}\arrow\commit{c9}\arrow\version{v0.2.0}\arrowup\arrowdown\arrow\commit{c12}\arrow\commit{c14}\arrow\version{v1.0.0}}
\put(0,40){\text{10ex}{Maint:}\phantom{\version{v0.1.0}\arrowup\arrowddown\arrow\commit{c6}\arrow\commit{c7}\arrow\commit{c9}\arrow\version{v0.2.0}\arrow}\commit{c11}\arrow\version{v0.2.1}}
\put(0,20){\text{10ex}{Maint-0.1:}\phantom{\version{v0.1.0}\arrow}\commit{c1}\arrow\commit{c3}\arrow\version{v0.1.1}\arrow\commit{c5}\arrow\version{v0.1.2}}
\end{picture}}%
\end{center}

\item From time to time (at least for every revision/maintenance
  release), merge from \emph{maint} into \emph{master}, and from
  \emph{master} into \emph{unstable}. Never merge in the opposite way.
\begin{center}
\fbox{\begin{picture}(250,80)\sffamily\scriptsize%
\put(0,0){Time:$\,\longmapsto$}
\put(0,60){\text{10ex}{Unstable:}\phantom{\version{v0.1.0}\arrow}\commit{c8}\arrowlen{31ex}\merge{+}\arrow\commit{c10}}
\put(0,40){\text{10ex}{Master:}\version{v0.1.0}\arrowup\arrowdown\arrow\commit{c6}\arrow\commit{c7}\arrowlen{15.5ex}\merge{+}\arrowup\arrow\commit{c9}\arrow\version{v0.2.0}}
\put(0,20){\text{10ex}{Maint:}\phantom{\version{v0.1.0}\arrow}\commit{c1}\arrow\commit{c3}\arrow\version{v0.1.1}\arrowup\arrow\commit{c5}\arrow\version{v0.1.2}}
\end{picture}}%
\end{center}

\item Each topic (bug-fix, feature, etc) will be tracked in an
  individual \emph{topic} branch, which will be forked off from the
  branch (\emph{maint} or \emph{master}) the topic will be eventually
  incorporated into (merged).
\begin{center}
\fbox{\begin{picture}(270,120)\sffamily\scriptsize%
\put(0,0){Time:$\,\longmapsto$}
\put(0,100){\text{10ex}{Unstable:}\phantom{\version{v0.1.0}\arrow}\commit{c8}\arrow\commit{c10}}
\put(0,80){\text{10ex}{Master:}\version{v0.1.0}\arrowup\arrowdown\arrow\commit{c6}\arrow\commit{c7}\arrow\commit{c9}\arrow\version{v0.2.0}}
\put(0,60){\text{10ex}{Maint:}\phantom{\version{v0.1.0}\arrow}\commit{c1}\arrowdown\arrowddown\arrowlen{12ex}\merge{+}\arrowlen{5ex}\merge{+}\arrow\version{v0.1.1}\arrow\commit{c5}\arrow\version{v0.1.2}}
\put(0,40){\text{10ex}{Bug-fix-1:}\phantom{\version{v0.1.0}\arrow\commit{c1}\arrow}\commit{c3}\arrowup}
\put(0,20){\text{10ex}{Bug-fix-2:}\phantom{\version{v0.1.0}\arrow\commit{c1}\arrow}\commit{c2}\arrow\commit{c4}\arrowdup}
\end{picture}}%
\end{center}

  Relevance changes will be incorporated from the topic branch into
  the \emph{unstable} branch (for stability testing) before being
  incorporated from the topic branch into the destination branch. Note
  that changes and bugfixes should be applied into the suitable
  \emph{topic} branch, and not into the \emph{unstable} branch. Never
  merge from the \emph{unstable} branch into the other ones.
\begin{center}
\fbox{\begin{picture}(270,100)\sffamily\scriptsize%
\put(0,0){Time:$\,\longmapsto$}
\put(0,80){\text{10ex}{Unstable:}\phantom{\version{v0.1.0}\arrow}\commit{c8}\arrow\commit{c10}\arrowlen{12ex}\merge{+}\arrowlen{5ex}\merge{+}\text{10ex}{testing}}
\put(0,60){\text{10ex}{Feature-1:}\phantom{\version{v0.1.0}\arrow\commit{c1}\arrow}\commit{c2}\arrow\commit{c4}\arrowup\arrow\commit{c12}\arrowup\arrowdown\text{20ex}{testing passed}}
\put(0,40){\text{10ex}{Master:}\version{v0.1.0}\arrowdup\arrowdown\arrow\commit{c6}\arrowup\arrow\commit{c7}\arrow\commit{c9}\arrowlen{12ex}\merge{+}\arrow\version{v0.2.0}}
\put(0,20){\text{10ex}{Maint:}\phantom{\version{v0.1.0}\arrow}\commit{c1}\arrow\commit{c3}\arrow\version{v0.1.1}\arrow\commit{c5}\arrow\version{v0.1.2}}
\end{picture}}%
\end{center}

\item As a general rule, changes should be split into small logical
  steps, and commit each of them separately. They should be
  consistent, working independently of any later commits and pass the
  test suite.
\end{itemize}%



%-------------------------------------------------------------------------------
%-------------------------------------------------------------------------------
%-------------------------------------------------------------------------------
%-------------------------------------------------------------------------------
%-------------------------------------------------------------------------------
\end{document}
%-------------------------------------------------------------------------------
